% Created 2015-01-19 lun 11:21
\documentclass[article, a4paper]{memoir}
\usepackage[utf8]{inputenc}
\usepackage[T1]{fontenc}
\usepackage{fixltx2e}
\usepackage{graphicx}
\usepackage{longtable}
\usepackage{float}
\usepackage{wrapfig}
\usepackage{rotating}
\usepackage[normalem]{ulem}
\usepackage{amsmath}
\usepackage{textcomp}
\usepackage{marvosym}
\usepackage{wasysym}
\usepackage{amssymb}
\usepackage{hyperref}
\tolerance=1000
\usepackage{color}
\usepackage{listings}
\usepackage{mathpazo}
\usepackage{memhfixc}
\usepackage{mempatch}
\usepackage{geometry}
\usepackage[usenames,dvipsnames]{xcolor}
\geometry{verbose,tmargin=2cm,bmargin=2cm,lmargin=2cm,rmargin=2cm}
\usepackage[spanish]{babel}
\hypersetup{bookmarks=true, breaklinks=true,pdftitle={Curriculum}, pdfauthor={Oscar Perpiñán Lamigueiro}, pdfnewwindow=true, colorlinks=true,linkcolor=Brown,citecolor=BrickRed, filecolor=black,urlcolor=Blue}
\renewcommand{\thesection}{\arabic{section}}
\renewcommand{\thesubsection}{\arabic{section}.\arabic{subsection}}
\author{Oscar Perpiñán Lamigueiro}
\date{19 de Enero de 2014}
\title{Curriculum Vitae}
\hypersetup{
  pdfkeywords={},
  pdfsubject={},
  pdfcreator={Emacs 24.4.1 (Org mode 8.2.7c)}}
\begin{document}

\maketitle


\section{Datos Personales}
\label{sec-1}

\subsection{Generales}
\label{sec-1-1}

\begin{itemize}
\item Nombre: Oscar Perpiñán Lamigueiro
\item Correo electrónico: \href{mailto:oscar.perpinan@upm.es}{oscar.perpinan@upm.es}
\item Web: \url{http://oscarperpinan.github.io}
\item LinkeIn: \url{http://es.linkedin.com/in/oscarperpinan}
\item twitter: \href{https://twitter.com/oscarperpinan}{@oscarperpinan}
\end{itemize}


\subsection{Contrato vigente en ámbito universitario}
\label{sec-1-2}

\begin{itemize}
\item Universidad Politécnica de Madrid
\begin{itemize}
\item Escuela Universitaria de Ingenería Técnica Industrial
\item Departamento de Ingeniería Eléctrica
\item Categoría Actual: Profesor Ayudante Doctor
\end{itemize}
\end{itemize}


\section{Títulos Académicos}
\label{sec-2}

\begin{itemize}
\item Doctor en Ingeniería Industrial
\begin{itemize}
\item Tesis: ``Grandes Centrales Fotovoltaicas: producción,
seguimiento y ciclo de vida'''
\item ETSII-UNED (Abr-2008)
\item Sobresaliente cum laude
\item Premio Extraordinario
\end{itemize}

\item Ingeniero de Telecomunicación
\begin{itemize}
\item ETSIT-UPM (Dic-1999)
\item Especialidad en Radiocomunicación
\end{itemize}
\end{itemize}


\section{Actividad Docente}
\label{sec-3}

\subsection{Profesor Ayudante Doctor en el Departamento de Ingeniería Eléctrica de la EUITI-UPM:}
\label{sec-3-1}

\begin{itemize}
\item Diseño Avanzado de Sistemas Energéticos Solares (cursos 2014/2015, 2013/2014 y 2012/2013).

\item Teoría de Circuitos (cursos 2014/2015, 2013/2014, 2012/2013 y 2011/2012)

\item Laboratorio de Teoría de Circuitos III (cursos 2014/2014 y 2013/2014)

\item Laboratorio de Teoría de Circuitos II (curso 2010/2011)

\item Fundamentos de Electrotecnia (Especialidad Eléctricos) (7,5 créditos) en los cursos 2008/2009 y 2009/2010.

\item Fundamentos de Electrotecnia (Especialidad Mecánicos) y Laboratorio de Fundamentos de Electrotecnia (Especialidad Mecánicos) (7,5 créditos) en los cursos 2008/2009 y 2009/2010.
\end{itemize}


\subsection{Profesor en la Escuela de Organización Industrial}
\label{sec-3-2}

Miembro del claustro del Máster de Energías Renovables y Mercado Energético (MERME) de la EOI:

\begin{itemize}
\item Cursos 2014-2015, 2013-2014, 2012-2013, 2011-2012, 2011-2010 y 2009-2010: Asignatura troncal ``Energía Solar Fotovoltaica''' con un total de 70 horas lectivas.

\item Curso 2009-2010: Asignatura troncal ``Energía Solar Fotovoltaica''' con un total de 40 horas lectivas y asignatura optativa ``Energía Solar Fotovoltaica Avanzada''' con un total de 40 horas lectivas.

\item Sesiones individuales con un total de 58 horas lectivas desde el curso 2001/2002 hasta el 2007/2008.

\item Dirección de Proyectos Fin de Máster:

\begin{itemize}
\item Curso 2013-2014: ``Electrificación Rural mediante Sistema Híbrido Eólico-Fotovoltáico en Roraima, Brasil''' y ``Electrificación Rural Aislada Fotovoltaica (ERAF) a Institutos de Telesecundaria en San Pedro Carchá, Alta Verapaz, Guatemala.'''
\item Curso 2011-2012: ``Servicios energéticos Renovables: E. Fotovoltaica, E. Mini-eólica y eficiencia energética en entornos urbanos'''
\item Curso 2010-2011: ``Comparativa y análisis de variabilidad espacio-temporal entre las medidas de radiación solar terrestres (SIAR) y satelitales (CM SAF). Estudio de productividad potencial'''.
\item Dirección de 13 Proyectos de Fin de Máster desde el curso 2001/2002 hasta el 2009/2010.
\end{itemize}
\end{itemize}


\subsection{Otros}
\label{sec-3-3}

\begin{itemize}
\item Supervisión on-site de Proyectos de Fin de Máster del European Máster in Renewable Energy de la Agencia EUREC en los cursos 2004/2005 y 2003/2004.
\end{itemize}


\section{Actividad Investigadora}
\label{sec-4}

\subsection{Pertenencia a Grupo de Investigación ``Sistemas Fotovoltaicos'''}
\label{sec-4-1}

Reconocido como consolidado por la Universidad Politécnica de
Madrid, con tres líneas de investigación: Sistemas fotovoltaicos
conectados a la red, Sistemas fotovoltaicos autónomos e híbridos y
Electrificación rural fotovoltaica.


\subsection{Dirección de Proyectos de Investigación}
\label{sec-4-2}
\begin{center}
\begin{tabular}{p{110mm}|p{17mm}|p{25mm}}
Título del Proyecto & Duración (meses) & Año Inicio\\
\hline
Insolación & 6 & 2008\\
Desarrollo de sistema inteligente de detección de fallos de funcionamiento de sistemas FV & 6 & 2008\\
Umbráculo móvil de 1 MWp & 18 & 2007\\
Desarrollo de una plataforma para la monitorización y seguimiento de sistemas fotovoltaicos & 12 & 2007\\
Mejora de la calidad del servicio energético en las aplicaciones de electrificación rural & 12 & 2005\\
Grandes centrales fotovoltaicas & 24 & 2004\\
\end{tabular}
\end{center}


\subsection{Participación en Proyectos de Investigación}
\label{sec-4-3}

\begin{longtable}{p{110mm}|p{17mm}|p{25mm}}
Título del Proyecto & Duración (meses) & Año Inicio\\
\hline
\endhead
\hline\multicolumn{3}{r}{Continued on next page} \\
\endfoot
\endlastfoot
PVCROPS. PhotoVoltaic Cost reduction, Reliability, Operational performance, Prediction and Simulation & 36 & 2012\\
ENERGOS. Optimización de la cargabilidad en líneas. & 48 & 2009\\
Caracterización de la variabilidad y comportamiento ante las perturbaciones de las plantas fotovoltaicas & 36 & 2007\\
Optimización del diseño eléctrico de módulos fotovoltaicos para minimizar las perdidas de potencia por dispersión y evitar los puntos calientes & 24 & 2007\\
Desarrollo y caracterización de tejados y fachadas fotovoltaicas ventiladas integradas en edificios & 12 & 2007\\
Conector de paneles dinámico CPD-1 y convertidor multipuente multipotencia CMM-1 & 12 & 2007\\
Desarrollo de seguidor de doble eje de gran tamaño para módulos de concentración & 24 & 2006\\
Moden$\backslash$$_{\text{II}}$ & 24 & 2006\\
Desarrollo de seguidor de doble eje de gran tamaño para módulos planos & 40 & 2005\\
Caracterización del comportamiento térmico de la fachada PVskin y su interacción con edificios modelo en clima mediterráneo & 12 & 2005\\
Sistema de desalinizacion mediante ósmosis inversa alimentado con energía solar fotovoltaica & 11 & 2005\\
TINA & 18 & 2004\\
Heliodomo: nuevo concepto de vivienda autosuficiente & 36 & 2004\\
Sevilla PV & 54 & 2003\\
PV generators integrated into sound barriers & 24 & 2001\\
PV grid connected system in a car parking & 24 & 2000\\
\end{longtable}


\subsection{Dirección de Tesis Doctorales}
\label{sec-4-4}

\begin{itemize}
\item Codirección de la Tesis Doctoral \guillemotleft{}Inserçao en Grande Escala de Geraçao Solar Fotovoltaica em Sistemas Elétricos de Potência\guillemotright{} de Marcelo Pinho Almeida, junto con el profesor Roberto Zilles, del Instituto de Energia e Ambiente de la Universidad de Sao Paulo.

\item Codirección de la Tesis Doctoral \guillemotleft{}New methodologies and improved models in the estimation of solar irradiation\guillemotright{} de Fernando Antoñanzas, junto con el profesor Francisco Javier Martínez de Ascacíbar, del grupo EDMANS de la Universidad de la Rioja.

\item Codirección de la Tesis Doctoral \guillemotleft{}Penetración de la Energía Fotovoltaica en el Sistema Eléctrico peninsular español. Condiciones del Mercado Eléctrico y Red de Transporte\guillemotright{} de José Melguizo, junto con el profesor Manuel Castro Gil, catedrático del DIEEC-ETSII (UNED).
\end{itemize}


\section{Publicaciones}
\label{sec-5}

\subsection{Libros}
\label{sec-5-1}

\begin{itemize}
\item Displaying time series, spatial and space-time data with R: stories of space and time
\begin{itemize}
\item Editorial: Chapman \& Hall/CRC
\item Publicación 4 de Abril de 2014.
\item ISBN: 9781466565203
\item \url{http://oscarperpinan.github.com/spacetime-vis/}
\end{itemize}

\item Energía Solar Fotovoltaica

\begin{itemize}
\item Libro principal en la asignatura ``Energía Solar Fotovoltaica''' del Máster de Energías Renovables y Mercado Energético de la EOI.
\item Publicado online con licencia Creative Commons.
\item \url{http://oscarperpinan.github.com/esf}
\end{itemize}

\item Diseño de Sistemas Fotovoltaicos

\begin{itemize}
\item Autores: Perpiñán O., Castro Gil M.A. y Colmenar A.
\item 1ª Edición, 2012
\item Editorial: Promotora General de Estudios, S.A.
\item ISBN: 978-84-95693-72-3
\end{itemize}

\item Sistemas de bombeo eólicos y fotovoltaicos

\begin{itemize}
\item Autor/es: Castro Gil, Manuel-Alonso y otros
\item Colección: Monografías de Energías Renovables
\item Editorial: Promotora General de Estudios, S.A.
\item ISBN: 84-95693-67-9
\end{itemize}

\item Energía eólica

\begin{itemize}
\item Autor/es: Colmenar Santos, Antonio y otros
\item Colección: Monografías de Energías Renovables
\item Publicación: Promotora General de Estudios, S.A.
\item ISBN: 84-86505-69-3
\end{itemize}
\end{itemize}


\subsection{Artículos y capítulos en libros}
\label{sec-5-2}
\subsubsection{Revistas Internacionales}
\label{sec-5-2-1}
Disponibles en \url{http://oscarperpinan.github.io/#papers}

\begin{itemize}
\item F. Antonanzas-Torres, Andres Sanz-Garcia, Javier Antonanzas-Torres, \textbf{Oscar Perpiñán} and Francisco Javier Martínez-de-Pisón-Ascacibar. \guillemotleft{}Current Status and Future Trends of the Evaluation of Solar Global Irradiation using Soft-Computing-Based Models.\guillemotright{} Soft Computing Applications for Renewable Energy and Energy Efficiency. IGI Global, 2015. 1-22. Web. 19 Jan. 2015. \url{10.4018/978-1-4666-6631-3.ch001}

\item F. Antonanzas-Torres, F.J. Martínez de Pisón, J. Antonanzas, \textbf{O. Perpiñán}, Downscaling of global solar irradiation in complex areas in R, Journal of Renewable and Sustainable Energy, 6, 063105 (2014), \href{http://dx.doi.org/10.1063/1.4901539}{10.1063/1.4901539}

\item F. Antonanzas-Torres, A. Sanz-Garcia, F. J. Martínez-de-Pisón, \textbf{O. Perpiñán}, J. Polo, Towards downscaling of aerosol gridded dataset for improving solar resource assessment. Application to Spain, Renewable Energy, Volume 71, November 2014, Pages 534-544, \href{http://dx.doi.org/10.1016/j.renene.2014.06.010}{10.1016/j.renene.2014.06.010}.

\item F. Antonanzas-Torres, A. Sanz-Garcia, F.J. Martínez-de-Pisón, \textbf{O. Perpiñán}, Evaluation and improvement of empirical models of global solar irradiation: Case study northern Spain, Renewable Energy, Volume 60, December 2013, Pages 604-614, ISSN 0960-1481, \href{http://dx.doi.org/10.1016/j.renene.2013.06.008}{10.1016/j.renene.2013.06.008}.

\item F. Antoñanzas, F. Cañizares, \textbf{O. Perpiñán}, Comparative assessment of global irradiation from a satellite estimate model (CM SAF) and on-ground measurements (SIAR): a Spanish case study, Renewable and Sustainable Energy Reviews, Volume 21, May 2013, Pages 248-261, \href{http://dx.doi.org/10.1016/j.rser.2012.12.033}{10.1016/j.rser.2012.12.033}.

\item \textbf{O. Perpiñán}, J. Marcos, E. Lorenzo, Electrical Power Fluctuations in a Network of DC/AC inverters in a Large PV Plant: relationship between correlation, distance and time scale, Solar Energy, Volume 88, February 2013, \href{http://dx.doi.org/10.1016/j.solener.2012.1}{10.1016/j.solener.2012.1}.

\item \textbf{O. Perpiñán}, M.A. Sánchez-Urán, F. Álvarez, J. Ortego, F. Garnacho, Signal analysis and feature generation for pattern identification of partial discharges in high-voltage equipment, Electric Power Systems Research, 2013, 95:C (56-65), \href{http://dx.doi.org/10.1016/j.epsr.2012.08.016}{10.1016/j.epsr.2012.08.016}.

\item \textbf{O. Perpiñán}, solaR: Solar Radiation and Photovoltaic Systems with R, Journal of Statistical Software, 2012. 50(9), (1-32).

\item \textbf{O. Perpiñán}, Cost of energy and mutual shadows in a two-axis tracking PV system, Renewable Energy, 2011, \url{10.1016/j.renene.2011.12.001}.

\item \textbf{O. Perpiñán} y E. Lorenzo, Analysis and synthesis of the variability of irradiance and PV power time series with the wavelet transform, Solar Energy, 85:1 (188-197), 2010, \url{10.1016/j.solener.2010.08.013}).

\item \textbf{O. Perpiñán}, Statistical analysis of the performance and simulation of a two-axis tracking PV system, Solar Energy, 83:11(2074–2085), 2009, \url{10.1016/j.solener.2009.08.008}.

\item \textbf{O. Perpiñán}, E. Lorenzo, M. A. Castro, y R. Eyras. Energy payback time of grid connected pv systems: comparison between tracking and fixed systems. Progress in Photovoltaics: Research and Applications, 17:137-147, 2009.

\item \textbf{O. Perpiñán}, E. Lorenzo, M. A. Castro, y R. Eyras. On the complexity of radiation models for PV energy production calculation. Solar Energy, 82:2 (125-131), 2008.

\item \textbf{O. Perpiñán}, E. Lorenzo, y M. A. Castro. On the calculation of energy produced by a PV grid-connected system. Progress in Photovoltaics: Research and Applications, 15(3):265–274, 2007.
\end{itemize}

\subsubsection{Revistas Nacionales}
\label{sec-5-2-2}

\begin{itemize}
\item Fernando Garnacho Vecino, Miguel Ángel Sánchez-Urán González, Javier
Ortego La Moneda, F. Alvarez, \textbf{O. Perpiñán},
Revisión periódica del estado del aislamiento de los cables de
AT mediante medidas de DPs on line, Energía: Ingeniería
energética y medioambiental, ISSN 0210-2056, Año nº 37, Nº
230, 2011, págs. 38-46.

\item \textbf{O. Perpiñán}, E. Lorenzo, M.A. Castro, Estimación de sombras mutuas
entre seguidores y optimización de separaciones, Era Solar, ISSN
0212-4157, Nº. 131, 2006 , págs. 28-37

\item \textbf{O. Perpiñán}, M.A. Castro, E. Lorenzo, Análisis y
comparación de funcionamiento de grandes plantas: photocampa y forum
Energía: Ingeniería energética y medioambiental, ISSN 0210-2056, Año
nº 32, Nº 190, 2006, págs. 63-68

\item J. Carretero, L. Mora-López, \textbf{O. Perpiñán}, A. Pereña, Mariano Sidrach
de Cardona Ortín, I. Martínez, M. Aritio, Parque tecnológico de
Andalucía: tecnología OPC. Monitorización wireless de una
instalación fotovoltaica de 56 kWp, Era solar: Energías renovables,
ISSN 0212-4157, Nº. 127, 2005, págs. 56-65.

\item Arancha Perpiñán Lamigueiro, \textbf{O. Perpiñán}, Elena Carmen
Horno, Bombeo de agua para riego con energía solar fotovoltáica:
sistemas de bombeo solar directo Riegos y drenajes XXI, ISSN
0213-3660, Nº 144, 2005, págs. 68-74

\item \textbf{O. Perpiñán}, R. Eyras, D. Jiménez, Antonio Gómez
Avilés-Casco, Sistemas fotovoltaicos en el Parque de las
Ciencias de Granada Era solar: Energías renovables, ISSN
0212-4157, Nº. 104, 2001, págs. 16-21
\end{itemize}


\section{Comunicaciones y Ponencias Presentadas a Congresos}
\label{sec-6}


\subsection{Congresos Internacionales}
\label{sec-6-1}
\begin{longtable}{p{87mm}|p{30mm}|p{15mm}|p{30mm}}
Titulo & Lugar & Fecha & Entidad Organizadora\\
\hline
\endhead
\hline\multicolumn{4}{r}{Continued on next page} \\
\endfoot
\endlastfoot
Downscaling of Solar Irradiation from Satellite Models & Logroño & Jul. 2013 & AEIPRO\\
New Procedure to Determine Insulation Condition of High Voltage Equipment by Means of \{PD\} Measurements in Service & Francia & 2012 & CIGRE\\
PD monitoring system of HV cable & Francia & Jun. 2011 & Jicable\\
PV solar tracking systems analysis & Italia & Sep. 2007 & WIP\\
A real case of building integrated PV. Isofoton offices in Malaga & Alemania & Sep. 2006 & WIP\\
Analysis and comparison of performance of large plants: Photocampa and Forum & Barcelona & Jun. 2005 & WIP\\
PV pumping systems: cases of study & Francia & Jun. 2004 & WIP\\
PV soundless- world record along the highway: a PV sound barrier with 500 kwp and ceramic based pv modules & Francia & Jun. 2004 & WIP\\
Forum solar: a large pergola for forum & Francia & Jun. 2004 & WIP\\
Architecture and PV: discussion and experiences & Francia & Jun. 2004 & WIP\\
PV pumping systems: cases of study & Tailandia & Ene. 2004 & PVSEC Comittee\\
Architectural integration of grid connected photovoltaic systems for schools in coslada & Japon & Mayo 2003 & WCPEC3\\
Photocampa: design and performance of the PV system & Japon & Mayo 2003 & WCPEC3\\
FIVE project-integration of pv systems on health emergency vehicles- results and conclusions & Italia & Oct. 2002 & WIP-ETA\\
PVSoundless: large PV sound barrier along a railway & Italia & Jun. 2002 & ISES\\
Integration of PV systems on health emergency vehicles. FIVE project & Alemania & Oct. 2001 & WIP\\
Photocampa: PV system integrated into a large car park & Alemania & Oct. 2001 & WIP\\
PV pergola for the chapel of men & Alemania & Oct. 2001 & WIP\\
Special module types for pv systems in high-profile buildings & Alemania & Oct. 2001 & WIP\\
\end{longtable}


\subsection{Congresos nacionales}
\label{sec-6-2}
\begin{longtable}{p{87mm}|p{30mm}|p{15mm}|p{30mm}}
Título & Lugar & Fecha & Entidad Organizadora\\
\hline
\endhead
\hline\multicolumn{4}{r}{Continued on next page} \\
\endfoot
\endlastfoot
meteoForecast: predicciones meteorológicas de modelos NWP en R & Santiago de Compostela & Oct. 2014 & Comunidad R-Hispano\\
Comparativa y análisis de variabilidad espacial entre medidas de radiación solar terrestre y satelital & Madrid & Nov. 2011 & AUR\\
solaR: geometría, radiación y energía solar en R & Madrid & Nov. 2011 & AUR\\
Datos geográficos de tipo raster en R & Madrid & Nov. 2011 & AUR\\
Instalación de energía solar en la nueva fabrica de Isofoton en el PTA de Málaga & Málaga & Jun. 2005 & AEIPRO\\
Solarizate: proyecto escuelas solares de Greenpeace-IDAE & Vigo & Sep. 2004 & Univ. Vigo\\
Sistema solar térmico y fotovoltaico en hotel urbano & Vigo & Sep. 2004 & Univ. Vigo\\
Centrales híbridas solar-diesel: nuestra experiencia & Vigo & Sep. 2004 & Univ. Vigo\\
Fachada doble fotovoltaica ``PVskin''': prototipos, investigación y desarrollo & Vigo & Sep. 2004 & Univ. Vigo\\
Experiencia en sistemas de bombeo solar y simulación matemática de bombeos solares con equipos estándar & Vigo & Sep. 2004 & Univ. Vigo\\
Monitorización wireless de instalación fotovoltaica de 56 kWp en el parque tecnológico de Andalucia basada en la tecnología OPC & Vigo & Sep. 2004 & Univ. Vigo\\
Instalación de energía solar térmica con maquina de absorción & Pamplona & Oct. 2003 & AEIPRO\\
Ósmosis inversa alimentada mediante energía solar fotovoltaica & Pamplona & Oct. 2003 & AEIPRO\\
Photocampa: sistema fotovoltaico integrado en estructura de aparcamiento & Barcelona & Oct. 2002 & AEIPRO\\
\end{longtable}



\section{Desarrollos}
\label{sec-7}

\begin{itemize}
\item \texttt{solaR} (\url{http://oscarperpinan.github.io/solar/}): paquete software basado en R compuesto por un conjunto de funciones destinadas al calculo de la radiación solar incidente en sistemas fotovoltaicos y a la simulacion del funcionamiento de diferentes aplicaciones de esta tecnologia energética.

\item \texttt{rasterVis} (\url{http://oscarperpinan.github.io/rastervis/}): paquete software basado en R para la visualización e interacción gráfica de datos espaciales masivos.

\item \texttt{meteoForecast} (\url{http://github.com/oscarperpinan/meteoForecast}): paquete software basado en R que permite obtener predicciones de modelos numéricos meteorológicos producidos por diferentes servicios en formato raster o como series temporales.

\item \texttt{PVF} (\url{https://github.com/iesiee/PVF}): paquete software basado en R que permite realizar predicciones de potencia producida por un sistema FV.

\item \texttt{pdCluster} (\url{http://pdcluster.r-forge.r-project.org/}): paquete software basado en R para la cuantificación, clasificación y análisis de importancia de variables de señales de descargas parciales en equipos de Alta Tensión.

\item \texttt{pxR} (\url{http://pxr.r-forge.r-project.org/}): paquete software basado en R para la manipulación de fuentes de datos basadas en el formato PC-Axis, habitualmente empleado por instituciones nacionales e internacionales para la publicación de información.
\end{itemize}



\section{Cursos y Seminarios Impartidos}
\label{sec-8}

\begin{itemize}
\item Curso \guillemotleft{}Introducción a R\guillemotright{} (8 horas) para profesores de la UPM (Diciembre 2014)

\item Curso \guillemotleft{}Introducción a R\guillemotright{} (10 horas) para investigadores del CEIGRAM-UPM (Noviembre 2014)

\item Taller \guillemotleft{}Visualización de Series Temporales\guillemotright{} en las VI Jornadas de Usuarios de R (Octubre 2014)

\item Taller \guillemotleft{}Visualización de Datos Raster\guillemotright{} en las VI Jornadas de Usuarios de R (Octubre 2014)

\item Ponencia \guillemotleft{}Data Visualization with R\guillemotright{} dentro del Máster \guillemotleft{}Data Driven Methods in Environmental Management and Conservation\guillemotright{} del Instituto
de Empresa (Febrero, 2013).

\item Curso \guillemotleft{}Introducción a R\guillemotright{} (8 horas) para profesores de la UNED (Marzo 2013).

\item Curso \guillemotleft{}R avanzado\guillemotright{} (8 horas) para profesores de la UNED (Marzo 2013).

\item Participacion en las ediciones 2014/2015, 2013/2014, 2012/2013, 2011/2012, 2010/2011 y 2010/2009 del Master propio de Energías Renovables y Medio Ambiente de la UPM, organizado por la EUITI-UPM, impartiendo el tema ``Diseño de plantas FV con seguimiento solar''' con una duración de 4,5 horas.

\item Curso ``Instalaciones de energía solar''', impartido del 15/09/10 al 16/10/11 con una duración de 109 horas, organizado por la ETSI-Montes-UPM, impartiendo el modulo ``Sistemas fotovoltaicos conectados a red''', con una duración de 5 horas.

\item Formación a distancia sobre Diseño y Optimización de Sistemas Fotovoltaicos al responsable de Sistemas Solares de la empresa MENA. Este proceso de formación, con una duración de 6 meses, se ha basado en las potencialidades del paquete software solaR, reseñado anteriormente.

\item Curso ``Técnico en energías renovables''', impartido del 01/09/09 al 27/10/09 con una duración de 200 horas, organizado por la EUITI-UPM, impartiendo el modulo ``Sistemas fotovoltaicos conectados a red''', con una duración de 5 horas.

\item Curso ``Instalaciones de energía solar''', impartido del 15/09/09 al 16/10/09 con una duración de 109 horas, organizado por la ETSI-Montes-UPM, impartiendo el modulo ``Sistemas fotovoltaicos conectados a red''', con una duración de 5 horas.

\item Curso ``Técnico en instalaciones solares en edificios''', impartido del 01/09/09 al 15/10/09, organizado por la EUITI-UPM, impartiendo el módulo ``Sistemas fotovoltaicos conectados a red''', con una duración de 25 horas.

\item Participacion en el curso ``Técnico en instalaciones fotovoltaicas y eolicas''', impartido del 06/10/09 al 04/12/09, organizado por la EUITI-UPM, impartiendo el módulo ``Sistemas fotovoltaicos conectados a red''', con una duración de 15 horas.

\item Participacion en el curso ``Energías renovables''', ediciones 2011/2012, 2010/2011 y 2009/2010, con una duración de 200 horas, organizado por la ETSI-Montes-UPM, impartiendo el módulo ``Energía solar fotovoltaica''', con una duración de 5 horas.
\end{itemize}

\section{Cursos y Seminarios recibidos}
\label{sec-9}

\emph{(Ordenados por duración)}

\begin{itemize}
\item Experto Universitario en Métodos Avanzados de Estadística
Aplicada (UNED, 2009/2010, 625 horas)

\item Aplicación de las energías renovables (ETSII-UPC, 2001/2002, 300
horas)

\item Caracterización de la radiación solar como recurso energético
(CIEMAT, 2006, 30 horas)

\item Prevención de riesgos laborales Baja Tensión y proximidad Alta
Tensión (CEFOIM, 2008, 30 horas)

\item Estadística en la investigación experimental (ICE-UPM, 2010, 28
horas)

\item Estadística comparativa y de investigación para uno y dos grupos
de muestras (ICE-UPM, 2009, 24 horas)

\item Habilidades de negociación (Criteria, 2004, 20 horas)

\item Curso eléctrico de Media Tensión (Pedro Giner Editorial, 2003,
18 horas)

\item Introduction to mathematical optimization techniques applied to
power systems generation operation planning (Universidad
Pontificia de Comillas, 2003, 20 horas)

\item Rechargeable batteries (OTTI Kolleg, 2002, 20 horas)

\item Wavelets en Estadística (ICE-UPM, 2010, 8 horas)
\end{itemize}


\section{Becas, Ayudas y Premios recibidos}
\label{sec-10}

\begin{itemize}
\item Premio Extraordinario de Doctorado.
\end{itemize}


\section{Actividad en Empresas y Profesión Libre}
\label{sec-11}


\begin{itemize}
\item Octubre 2010- Enero 2009: Ejercicio libre de la profesión:
consultoría sobre diseño y análisis de funcionamiento de
sistemas fotovoltaicos.

\item Diciembre 2008- Enero 2007: Subdirector Técnico de ISOFOTON

\begin{itemize}
\item Responsable de las Áreas de I+D+i, Producto BOS y Difusión
Técnica
\item Equipo compuesto por 10 personas.
\end{itemize}
\end{itemize}


\begin{itemize}
\item Enero 2007-Mayo 2002: Gerente de Ingeniería (Dpto. Técnico de
ISOFOTON)

\begin{itemize}
\item Responsable de ofertas técnicas, diseño de proyectos y
proyectos de ejecución.
\item Equipo compuesto por 7 personas.
\end{itemize}
\end{itemize}


\begin{itemize}
\item Mayo 2002-Noviembre 2001: Gerente de Instalaciones
(Dpto. Técnico de ISOFOTON)

\begin{itemize}
\item Responsable de gestión y dirección de proyectos, y jefatura de
obras.
\item Equipo compuesto por 7 personas.
\end{itemize}
\end{itemize}


\begin{itemize}
\item Noviembre 2001-Marzo 2000: Ingeniero de Proyectos (Dpto. Técnico
de ISOFOTON)
\end{itemize}



\section{Otros Méritos Docentes o de Investigación}
\label{sec-12}

\subsection{Revisor para revistas}
\label{sec-12-1}

\begin{itemize}
\item \emph{Solar Energy}

\item \emph{Journal of Statistical Software}

\item \emph{Computers and Geosciences}

\item \emph{Applied Energy}

\item \emph{Journal of Solar Engineering}

\item \emph{IET Renewable Power Generation}
\end{itemize}

\subsection{Miembro de tribunales de Tesis Doctoral}
\label{sec-12-2}

\begin{itemize}
\item ``El Proyecto Pierre Auger como Red de Sistemas Fotovoltaicos Aislados de Alta Estadística''' (Iago Rodriguez Cabo, USC 2015)

\item ``Predicción espacio-temporal de la irradiancia solar global a corto plazo en España mediante geoestadística y redes neuronales artificiales''' (Federico Vladimir Gutierrez Corea, UPM 2014)

\item ``Energía solar fotovoltaica: competitividad y evaluación económica, comparativa y modelos''' (Eduardo Collado Fernández, UNED 2009)
\end{itemize}

\subsection{Acreditaciones}
\label{sec-12-3}

\begin{itemize}
\item Acreditación de la Agencia Nacional de Evaluación de la Calidad y Acreditación (ANECA) para la figura de Profesor Titular de Universidad.

\item Evaluación positiva por la Agencia Nacional de Evaluación de la Calidad y Acreditación (ANECA) para la figura de Profesor Contratado Doctor, Profesor Ayudante Doctor y Profesor de Universidad Privada.

\item Evaluación positiva por la Agencia de Calidad, Acreditación y Prospectiva de las Universidades de Madrid (ACAP) para la figura de Profesor Contratado Doctor y Profesor Ayudante Doctor.

\item Evaluación favorable de la actividad docente según el programa DOCENTIA de la ANECA durante el período comprendido entre noviembre de 2008 y junio de 2011.
\end{itemize}

\subsection{Asociaciones}
\label{sec-12-4}

\begin{itemize}
\item Presidente del Comité Organizador y miembro del Comité Científico de las III Jornadas de Usuarios de R (\url{http://r-es.org/III+Jornadas}).

\item Miembro del Comité Científico de las IV, V, y VI Jornadas de Usuarios de R (\url{http://r-es.org/IV+Jornadas}, \url{http://r-es.org/V+Jornadas}, \url{http://r-es.org/VI+Jornadas}).

\item Vocal de la Asociación de Usuarios de R.

\item Miembro de la Comisión Técnica de la Asociación de la Industria Fotovoltaica (ASIF) hasta Diciembre del 2008.

\item Miembro del grupo GT C del Comité de Normalización SC82 hasta Diciembre del 2008.
\end{itemize}
% Emacs 24.4.1 (Org mode 8.2.7c)
\end{document}