% Created 2018-06-29 vie 11:36
% Intended LaTeX compiler: pdflatex
\documentclass[article, a4paper]{memoir}
\usepackage[utf8]{inputenc}
\usepackage[T1]{fontenc}
\usepackage{graphicx}
\usepackage{grffile}
\usepackage{longtable}
\usepackage{wrapfig}
\usepackage{rotating}
\usepackage[normalem]{ulem}
\usepackage{amsmath}
\usepackage{textcomp}
\usepackage{amssymb}
\usepackage{capt-of}
\usepackage{hyperref}
\usepackage{color}
\usepackage{listings}
\usepackage{mathpazo}
\usepackage{memhfixc}
\usepackage{geometry}
\usepackage[usenames,dvipsnames]{xcolor}
\geometry{verbose,tmargin=2cm,bmargin=2cm,lmargin=2cm,rmargin=2cm}
\usepackage[spanish]{babel}
\usepackage{enumitem}
\setlist{itemsep=-2pt}
\hypersetup{bookmarks=true, breaklinks=true,pdftitle={Curriculum}, pdfauthor={Oscar Perpiñán Lamigueiro}, pdfnewwindow=true, colorlinks=true,linkcolor=Brown,citecolor=BrickRed, filecolor=black,urlcolor=Blue}
\counterwithout{section}{chapter}
\setsecheadstyle{\Large\scshape\bfseries\raggedright}
\setsubsubsecheadstyle{\itshape\raggedright}
\author{Oscar Perpiñán Lamigueiro}
\date{Junio, 2018}
\title{Curriculum Vitae}
\hypersetup{
 pdfauthor={Oscar Perpiñán Lamigueiro},
 pdftitle={Curriculum Vitae},
 pdfkeywords={},
 pdfsubject={},
 pdfcreator={Emacs 25.2.2 (Org mode 9.1.13)}, 
 pdflang={Spanish}}
\begin{document}

\maketitle

\section{Datos Personales}
\label{sec:orgbb2edf2}
\subsection{Generales}
\label{sec:orgae8e558}

\begin{itemize}
\item Nombre: Oscar Perpiñán Lamigueiro
\item Correo electrónico: \href{mailto:oscar.perpinan@upm.es}{oscar.perpinan@upm.es}
\item Web: \url{http://oscarperpinan.github.io}
\item LinkedIn: \url{http://es.linkedin.com/in/oscarperpinan}
\end{itemize}

\subsection{Contrato vigente en ámbito universitario}
\label{sec:orgf434f4d}

\begin{itemize}
\item Universidad Politécnica de Madrid
\begin{itemize}
\item Escuela Técnica Superior de Ingeniería de Diseño Industrial
\item Departamento de Ingeniería Eléctrica, Electrónica, Automática, y de Física Aplicada
\item Categoría Actual: Profesor Contratado Doctor
\end{itemize}
\end{itemize}


\section{Títulos Académicos}
\label{sec:org17b1bbf}
\subsection{Doctor}
\label{sec:org0a3b8f5}
\begin{itemize}
\item ETSII-UNED (Abr-2008)
\item Programa de Doctorado \guillemotleft{}Sistemas de Ingeniería Eléctrica, Electrónica y de Control\guillemotright{}
\item Tesis ``Grandes Centrales Fotovoltaicas: producción, seguimiento y ciclo de vida'' calificada con Sobresaliente cum laude y Premio Extraordinario.
\end{itemize}

\subsection{Ingeniero de Telecomunicación}
\label{sec:org7d07fda}
\begin{itemize}
\item ETSIT-UPM (Dic-1999)
\item Especialidad en Radiocomunicación
\item Proyecto de Fin de Carrera: ``Desarrollo de un protocolo de medida de lámparas fluorescentes para instalaciones fotovoltaicas de electrificación rural''
\end{itemize}

\newpage

\section{Puestos Docentes Desempeñados}
\label{sec:org17b7e85}
\begin{itemize}
\item Subdirector de Ordenación Académica - Jefe de Estudios de la ETSIDI-UPM
\begin{itemize}
\item Fecha de Nombramiento: Octubre 2015
\item Fecha de Cese: Mayo 2018
\end{itemize}
\end{itemize}

\subsection{Profesor Contratado Doctor}
\label{sec:org841a495}
\begin{itemize}
\item Centro: Escuela Técnica Superior de Ingeniería y Diseño Industrial, Universidad Politécnica de Madrid.
\item Régimen de Dedicación: Completa.
\item Actividad: Docencia e Investigación
\item Fecha Nombramiento: 4 de Octubre de 2015.
\end{itemize}

\subsection{Profesor Ayudante Doctor}
\label{sec:orgc5207ba}
\begin{itemize}
\item Centro: Escuela Técnica Superior de Ingeniería y Diseño Industrial, Universidad Politécnica de Madrid.
\item Régimen de Dedicación: Completa.
\item Actividad: Docencia e Investigación
\item Fecha Nombramiento: 4 de Octubre de 2010.
\item Fecha Cese: 3 de Octubre de 2015.
\end{itemize}

\subsection{Profesor Asociado}
\label{sec:orgb3ae4e0}
\begin{itemize}
\item Centro: Escuela Universitaria de Ingeniería Técnica Industrial, Universidad Politécnica de Madrid.
\item Régimen de Dedicación: Parcial (6+6).
\item Actividad: Docencia
\item Fecha Nombramiento: 14 de Noviembre de 2008
\item Fecha Cese: 30 de Septiembre de 2010
\end{itemize}

\newpage

\section{Actividad Docente}
\label{sec:orgf96af65}
\subsection{Escuela Técnica Superior de Ingeniería y Diseño Industrial}
\label{sec:org882902f}
\begin{center}
\begin{tabular}{p{18mm}p{45mm}p{40mm}p{10mm}p{15mm}p{10mm}}
Año & Asignatura & Titulación & Curso Titulación & Carácter & Horas\\
\hline
2008-2009 & Fund. de Electrotecnia & Ing. Técnico Industrial (Esp. Eléctricos) & 1 & Teoría & 75\\
2008-2009 & Fund. de Electrotecnia & Ing. Técnico Industrial (Esp. Mecánicos) & 1 & Teoría y Práctica & 75\\
\hline
2009-2010 & Fund. de Electrotecnia & Ing. Técnico Industrial (Esp. Eléctricos) & 1 & Teoría & 75\\
2009-2010 & Fund. de Electrotecnia & Ing. Técnico Industrial (Esp. Mecánicos) & 1 & Teoría y Práctica & 75\\
\hline
2010-2011 & Teoría de Circuitos II & Ing. Técnico Industrial (Esp. Eléctricos) & 2 & Práctica & 75\\
2010-2011 & Electrometría & Ing. Técnico Industrial (Esp. Eléctricos) & 1 & Práctica & 30\\
2010-2011 & Teoría de Circuitos I & Grado en Ing. Eléctrica & 1 & Práctica & 45\\
\hline
2011-2012 & Teoría de Circuitos & Grado en Ing. Química & 2 & Práctica & 15\\
2011-2012 & Teoría de Circuitos & Grado en Ing. Electrónica Industrial y Automática & 2 & Teoría & 45\\
2011-2012 & Teoría de Circuitos & Grado en Ing. Electrónica Industrial y Automática & 2 & Práctica & 15\\
2011-2012 & Teoría de Circuitos & Grado en Ing. Eléctrica & 2 & Práctica & 15\\
2011-2012 & Teoría de Circuitos & Grado en Ing. en Diseño Industrial & 2 & Teoría y Práctica & 60\\
2011-2012 & Teoría de Circuitos & Grado en Ing. Mecánica & 2 & Práctica & 15\\
\hline
2012-2013 & Teoría de Circuitos III & Grado en Ing. Eléctrica & 3 & Práctica & 15\\
2012-2013 & Teoría de Circuitos & Grado en Ing. Química & 2 & Teoría & 41\\
2012-2013 & Teoría de Circuitos & Grado en Ing. Electrónica Industrial y Automática & 2 & Práctica & 15\\
2012-2013 & Diseño Avanzado de Sistemas de Energía Solar & Máster Universitario en Ingeniería de la Energía & 1 & Teoría y Práctica & 10.5\\
2012-2013 & Teoría de Circuitos & Grado en Ing. en Diseño Industrial & 2 & Teoría & 41\\
2012-2013 & Teoría de Circuitos & Grado en Ing. Mecánica & 2 & Práctica & 30\\
\hline
2013-2014 & Teoría de Circuitos III & Grado en Ing. Eléctrica & 3 & Práctica & 64\\
2013-2014 & Teoría de Circuitos & Grado en Ing. Electrónica Industrial y Automática & 2 & Teoría y Práctica & 83\\
2013-2014 & Diseño Avanzado de Sistemas de Energía Solar & Máster Universitario en Ingeniería de la Energía & 1 & Teoría y Práctica & 10.5\\
2013-2014 & Teoría de Circuitos & Grado en Ing. Mecánica & 2 & Teoría y Práctica & 67\\
2013-2014 & Teoría de Circuitos & Grado en Ing. de Diseño Industrial & 2 & Teoría y Práctica & 48\\
\hline
2014-2015 & Teoría de Circuitos III & Grado en Ing. Eléctrica & 3 & Práctica & 64\\
2014-2015 & Teoría de Circuitos & Grado en Ing. Electrónica Industrial y Automática & 2 & Teoría y Práctica & 83\\
2014-2015 & Diseño Avanzado de Sistemas de Energía Solar & Máster Universitario en Ingeniería de la Energía & 1 & Teoría y Práctica & 10.5\\
2014-2015 & Teoría de Circuitos & Grado en Ing. Mecánica & 2 & Teoría y Práctica & 67\\
2014-2015 & Teoría de Circuitos & Grado en Ing. de Diseño Industrial & 2 & Teoría y Práctica & 48\\
\hline
2015-2016 & Teoría de Circuitos & Grado en Ing. Electrónica & 2 & Teoría y Práctica & 48\\
2015-2016 & Electrónica & Grado en Ing. Química & 3 & Teoría y Práctica & 48\\
2015-2016 & Teoría de Circuitos & Grado en Ing. Mecánica & 2 & Teoría y Práctica & 48\\
\hline
2016-2017 & Teoría de Circuitos & Grado en Ing. Electrónica & 2 & Teoría y Práctica & 48\\
2016-2017 & Electrónica & Grado en Ing. Química & 3 & Teoría y Práctica & 48\\
2016-2017 & Electrónica & Grado en Ing. Electrónica & 2 & Teoría y Práctica & 10\\
2016-2017 & Teoría de Circuitos & Grado en Ing. Mecánica & 2 & Teoría y Práctica & 48\\
2016-2017 & Informática & Grado en Ing. Química & 1 & Teoría y Práctica & 60\\
\hline
2017-2018 & Teoría de Circuitos & Grado en Ing. Electrónica & 2 & Teoría y Práctica & 48\\
2017-2018 & Teoría de Circuitos & Grado en Ing. Electrónica & 2 & Laboratorio & 12\\
2017-2018 & Teoría de Circuitos & Grado en Ing. Mecánica & 2 & Teoría y Práctica & 48\\
2017-2018 & Informática & Grado en Ing. Química & 1 & Teoría y Práctica & 60\\
\hline
\end{tabular}
\end{center}

\subsection{Escuela de Organización Industrial}
\label{sec:orgc9445ce}

Colaborador en el Máster de Energías Renovables y Mercado Energético (MERME) de la Escuela de Organización Industrial:

\begin{itemize}
\item Desde 2009-2010 a la actualidad: Asignatura troncal ``Energía Solar Fotovoltaica'' con un total de 70 horas lectivas.

\item Curso 2008-2009: Asignatura troncal ``Energía Solar Fotovoltaica'' con un total de 40 horas lectivas y asignatura optativa ``Energía Solar Fotovoltaica Avanzada'' con un total de 40 horas lectivas.

\item Sesiones individuales con un total de 58 horas lectivas desde el curso 2001/2002 hasta el 2007/2008.
\end{itemize}


\section{Actividad Investigadora}
\label{sec:org66c8584}
\subsection{Dirección de Tesis Doctorales}
\label{sec:orgd81e994}

\begin{itemize}
\item Codirección de la Tesis Doctoral \guillemotleft{}Variabilidad espacio-temporal del recurso solar fotovoltaico en Europa y el Mediterráneo\guillemotright{} de Claudia Gutiérrez Escribano, junto con el profesor Miguel Ángel Gaertner, de la Universidad de Castilla-La Mancha. En elaboración.

\item Codirección de la Tesis Doctoral \guillemotleft{}Inserçao en Grande Escala de Geraçao Solar Fotovoltaica em Sistemas Elétricos de Potência\guillemotright{} de Marcelo Pinho Almeida, junto con el profesor Roberto Zilles, del Instituto de Energia e Ambiente de la Universidad de Sao Paulo. Mayo de 2017. Sobresaliente cum laude.

\item Codirección de la Tesis Doctoral \guillemotleft{}New methodologies and improved models in the estimation of solar irradiation\guillemotright{} de Fernando Antoñanzas, junto con el profesor Francisco Javier Martínez de Ascacíbar, del grupo EDMANS de la Universidad de la Rioja. Abril 2016. Sobresaliente cum laude.
\end{itemize}

\subsection{Dirección de Trabajos Tutelados}
\label{sec:org0675df1}
\subsubsection{ETSIDI}
\label{sec:org5850561}
\begin{itemize}
\item \guillemotleft{}Estudio técnico-económico de una instalación fotovoltaica para autoconsumo en un edificio de oficinas en Madrid.\guillemotright{}, Marta Mejías, Grado en Ingeniería Eléctrica, 2016-2017.
\item \guillemotleft{}Proyecto tipo de instalación fotovoltaica en edificio industrial o de servicios.\guillemotright{}, Jorge Crespo, Grado en Ingeniería Eléctrica, 2016-2017.
\item \guillemotleft{}Interfaz gráfica para solaR con shiny\guillemotright{}, Gonzalo Mellizo-Soto, Grado en Ingeniería Eléctrica, 2016-2017.
\item \guillemotleft{}Estudio de Modelos para la estimación de la Radiación Solar Global en Ecuador\guillemotright{}, Eliana Carolina Ormeño Mejía, Máster en Ingeniería de la Energía, 2015-2016.
\item \guillemotleft{}Corrección de la medida de radiación solar difusa de la ETSIDI\guillemotright{}, Tomás Bobillo, Máster en Ingeniería de la Energía, 2015-2016.
\item \guillemotleft{}Estimation de la puissance des systèmes photovoltaïques avec Machine Learning\guillemotright{}, Imrane Dhmani (Université Montpellier 2). 2013-2014.
\item \guillemotleft{}Une estimation de puissance dans un système photovoltaique avec Random Forest\guillemotright{}, Kobaa Abdeslam (Université Montpellier 2). 2013-2014.
\end{itemize}

\subsubsection{Escuela de Organización Industrial}
\label{sec:orgc21655e}
\begin{itemize}
\item 2013-2014:
\begin{itemize}
\item ``Electrificación Rural mediante Sistema Híbrido Eólico-Fotovoltaico en Roraima, Brasil''
\item ``Electrificación Rural Aislada Fotovoltaica (ERAF) a Institutos de Telesecundaria en San Pedro Carchá, Alta Verapaz, Guatemala.''
\end{itemize}
\item 2011-2012:
\begin{itemize}
\item ``Servicios energéticos Renovables: E. Fotovoltaica, E. Mini-eólica y eficiencia energética en entornos urbanos''
\end{itemize}
\item 2010-2011:
\begin{itemize}
\item ``Comparativa y análisis de variabilidad espacio-temporal entre las medidas de radiación solar terrestres (SIAR) y satelitales (CM SAF). Estudio de productividad potencial''.
\end{itemize}
\item 2008/2009:
\begin{itemize}
\item \guillemotleft{}La integración de la generación distribuida de origen fotovoltaico con la red de distribución actual.
\end{itemize}
\item 2007/2008:
\begin{itemize}
\item \guillemotleft{}Proyecto de Instalación Fotovoltaica de 100 kW para la conexión a red de BT. Sistemas Híbridos Fotovoltaico-Diesel\guillemotright{}
\end{itemize}
\item 2006/2007:
\begin{itemize}
\item \guillemotleft{}Instalación solar Polígono Albresa\guillemotright{}
\item \guillemotleft{}Parque solar FV en Huelva\guillemotright{}
\item \guillemotleft{}Instalación solar en Jarandilla de la Vera\guillemotright{}
\end{itemize}
\item 2005/2006:
\begin{itemize}
\item \guillemotleft{}Planta seguimiento BT: estudio de posibilidades en mercado libre\guillemotright{}
\end{itemize}
\item 2004/2005
\begin{itemize}
\item \guillemotleft{}Sistema fotovoltaico conectado a red, Fontiveros (Ávila)\guillemotright{}
\item \guillemotleft{}Instalación fotovoltaica conectada a red\guillemotright{}
\end{itemize}
\item 2003/2004:
\begin{itemize}
\item \guillemotleft{}Integración de un sistema de energía solar en un centro escolar\guillemotright{}
\end{itemize}
\item 2002/2003:
\begin{itemize}
\item \guillemotleft{}Sistema fotovoltaico conectado a red integrado en pérgolas del Ayuntamiento de Tres Cantos\guillemotright{}
\end{itemize}
\item 2001/2002:
\begin{itemize}
\item \guillemotleft{}PVSOUNDLESS: Sistema fotovoltaico integrado en una barrera acústica\guillemotright{}
\end{itemize}
\end{itemize}

\subsubsection{EUREC}
\label{sec:org183b089}
Supervisión on-site de Proyectos de Fin de Máster del European Master in Renewable Energy de la Agencia EUREC en los cursos 2004/2005 y 2003/2004.

\section{Libros y Capítulos en Libros}
\label{sec:org42b5f0e}
\subsection{Displaying time series, spatial and space-time data with R}
\label{sec:org18dd9fd}
\begin{itemize}
\item Segunda Edición, O. Perpiñán, Chapman and Hall/CRC, 2018, ISBN 9781138089983. \url{http://oscarperpinan.github.io/bookvis/}
\item Primera Edición: O. Perpiñán, Chapman and Hall/CRC, 2014, ISBN 9781466565203. \url{http://oscarperpinan.github.com/spacetime-vis/}
\end{itemize}
\subsection{Energía Solar Fotovoltaica}
\label{sec:org11bad46}
O. Perpiñán, 2018. Libro autoeditado, publicado online con licencia Creative Commons.  \url{http://oscarperpinan.github.com/esf}

\subsection{Diseño de Sistemas Fotovoltaicos}
\label{sec:orgb71e70e}
\textbf{O. Perpiñán}, M.A. Castro Gil, A. Colmenar, Promotora General de Estudios, S.A., 2012, ISBN: 978-84-95693-72-3

\subsection{Soft Computing Applications for Renewable Energy and Energy Efficiency}
\label{sec:org9c4dc19}
F. Antonanzas-Torres, A. Sanz-Garcia, J. Antonanzas, \textbf{O. Perpiñán}, F.J. Martínez-de-Pisón, Current Status and Future Trends of the Evaluation of Solar Global Irradiation using Soft-Computing-Based Models, en Soft Computing Applications for Renewable Energy and Energy Efficiency. M. Cascales, M. Sánchez-Lozano, A.D. Masegosa, C. Cruz,  Series in Advances in Environmental Engineering and Green Technologies, IGI Global, 2015, (1-22) ISBN 9781466666320. \href{http://dx.doi.org/10.4018/978-1-4666-6631-3.ch001}{10.4018/978-1-4666-6631-3.ch001}.

\subsection{Sistemas de bombeo eólicos y fotovoltaicos}
\label{sec:org33c096f}
M. Castro,A. Colmenar, R.P. Fiffe, M. Pérez, \textbf{O. Perpiñán}, J.M. Perulero. Monografías de Energías Renovables, Promotora General de Estudios, S.A., 2011, ISBN 84-95693-67-9

\subsection{Energía eólica}
\label{sec:org8827a84}
M. Castro, A. Colmenar, \textbf{O. Perpiñán}, C. Sánchez Naranjo. Monografías de Energías Renovables, Promotora General de Estudios, S.A., 2011, ISBN 84-86505-69-3.

\section{Artículos en Revistas}
\label{sec:org26bcbbf}
\subsection{Publicaciones Internacionales}
\label{sec:orga8bca12}
Disponibles en \url{http://oscarperpinan.github.io/\#papers}

\begin{enumerate}
\item C. Gutiérrez, M. A. Gaertner, \textbf{O. Perpiñán}, C. Gallardo, E. Sánchez, A multi-step scheme for spatial analysis of solar and photovoltaic production variability and complementarity, Solar Energy, Volume 158, 2017, Pages 100-116, ISSN 0038-092X,  \href{https://doi.org/10.1016/j.solener.2017.09.037}{10.1016/j.solener.2017.09.037}.

\item M. Pinho Almeida, M. Muñoz, I. de la Parra, \textbf{O. Perpiñán}, Comparative study of PV power forecast using parametric and nonparametric PV models, Solar Energy, 155, 2017: 854-866, ISSN 0038-092X, \href{https://doi.org/10.1016/j.solener.2017.07.032}{10.1016/j.solener.2017.07.032}.

\item J. Muñoz, \textbf{O. Perpiñán}. A Simple Model for the Prediction of Yearly Energy Yields for Grid-Connected PV Systems Starting from Monthly Meteorological Data. Renewable Energy 97, 2016: 680–88. \href{http://dx.doi.org/10.1016/j.renene.2016.06.023}{10.1016/j.renene.2016.06.023}.

\item M. Pinho Almeida, \textbf{O. Perpiñán}, L. Narvarte, PV Power Forecast Using a Nonparametric PV Model. Solar Energy 115 (May 2015): 354–68. \href{http://dx.doi.org/10.1016/j.solener.2015.03.006}{10.1016/j.solener.2015.03.006}.$\backslash$\ (Índice de impacto: 3.868, Tercil T1 en categoría \emph{Energy \& Fuels})

\item F. Antonanzas-Torres, F.J. Martínez de Pisón, J. Antonanzas, \textbf{O. Perpiñán}, Downscaling of global solar irradiation in complex areas in R, Journal of Renewable and Sustainable Energy, 6, 063105 (2014), \href{http://dx.doi.org/10.1063/1.4901539}{10.1063/1.4901539} $\backslash$\ (Índice de impacto: 0.925, Tercil T3 en categoría \emph{Energy \& Fuels})

\item F. Antonanzas-Torres, A. Sanz-Garcia, F. J. Martínez-de-Pisón, \textbf{O. Perpiñán}, J. Polo, Towards downscaling of aerosol gridded dataset for improving solar resource assessment. Application to Spain, Renewable Energy, Volume 71, November 2014, Pages 534-544, \href{http://dx.doi.org/10.1016/j.renene.2014.06.010}{10.1016/j.renene.2014.06.010}. $\backslash$\ (Índice de impacto: 3.85, Tercil T1 en categoría \emph{Energy \& Fuels})

\item F. Antonanzas-Torres, A. Sanz-Garcia, F.J. Martínez-de-Pisón, \textbf{O. Perpiñán}, Evaluation and improvement of empirical models of global solar irradiation: Case study northern Spain, Renewable Energy, Volume 60, December 2013, Pages 604-614, ISSN 0960-1481, \href{http://dx.doi.org/10.1016/j.renene.2013.06.008}{10.1016/j.renene.2013.06.008}. $\backslash$\ (Índice de impacto: 2.99, Tercil T1 en categoría \emph{Energy \& Fuels})

\item F. Antoñanzas, F. Cañizares, \textbf{O. Perpiñán}, Comparative assessment of global irradiation from a satellite estimate model (CM SAF) and on-ground measurements (SIAR): a Spanish case study, Renewable and Sustainable Energy Reviews, Volume 21, May 2013, Pages 248-261, \href{http://dx.doi.org/10.1016/j.rser.2012.12.033}{10.1016/j.rser.2012.12.033}. $\backslash$\ (Índice de impacto: 5.51, Tercil T1 en categoría \emph{Energy \& Fuels})

\item \textbf{O. Perpiñán}, J. Marcos, E. Lorenzo, Electrical Power Fluctuations in a Network of DC/AC inverters in a Large PV Plant: relationship between correlation, distance and time scale, Solar Energy, Volume 88, February 2013, \href{http://dx.doi.org/10.1016/j.solener.2012.12.004}{10.1016/j.solener.2012.004}. $\backslash$\ (Índice de impacto: 3.541, Tercil T1 en categoría \emph{Energy \& Fuels})

\item \textbf{O. Perpiñán}, M.A. Sánchez-Urán, F. Álvarez, J. Ortego, F. Garnacho, Signal analysis and feature generation for pattern identification of partial discharges in high-voltage equipment, Electric Power Systems Research, 2013, 95:C (56-65), \href{http://dx.doi.org/10.1016/j.epsr.2012.08.016}{10.1016/j.epsr.2012.08.016}.$\backslash$\ (Índice de impacto: 1.69, Tercil T1 en categoría \emph{Engineering, Electrical \& Electronic})

\item \textbf{O. Perpiñán}, solaR: Solar Radiation and Photovoltaic Systems with R, Journal of Statistical Software, 2012. 50(9), (1-32). $\backslash$\ (Índice de impacto: 4.91, Tercil T1 en categoría \emph{Statistics \& Probability})

\item \textbf{O. Perpiñán}, Cost of energy and mutual shadows in a two-axis tracking PV system, Renewable Energy, 2012, \href{http://dx.doi.org/10.1016/j.renene.2011.12.001}{10.1016/j.renene.2011.12.001}. $\backslash$\ (Índice de impacto: 2.98, Tercil T1 en categoría \emph{Energy \& Fuels})

\item \textbf{O. Perpiñán} y E. Lorenzo, Analysis and synthesis of the variability of irradiance and PV power time series with the wavelet transform, Solar Energy, 85:1 (188-197), 2011, \href{http://dx.doi.org/10.1016/j.solener.2010.08.013}{10.1016/j.solener.2010.08.013}).$\backslash$\ (Índice de impacto: 2.48, Tercil T1 en categoría \emph{Energy \& Fuels})

\item \textbf{O. Perpiñán}, Statistical analysis of the performance and simulation of a two-axis tracking PV system, Solar Energy, 83:11(2074–2085), 2009, \href{http://dx.doi.org/10.1016/j.solener.2009.08.008}{10.1016/j.solener.2009.08.008}. $\backslash$\ (Índice de impacto: 2.01, Tercil T2 en categoría \emph{Energy \& Fuels})

\item \textbf{O. Perpiñán}, E. Lorenzo, M. A. Castro, y R. Eyras. Energy payback time of grid connected PV systems: comparison between tracking and fixed systems. Progress in Photovoltaics: Research and Applications, 17:137-147, 2009.$\backslash$\ (Índice de impacto: 4.7; Tercil T1 en categoría \emph{Energy \& Fuels})

\item \textbf{O. Perpiñán}, E. Lorenzo, M. A. Castro, y R. Eyras. On the complexity of radiation models for PV energy production calculation. Solar Energy, 82:2 (125-131), 2008$\backslash$\ (Índice de impacto: 1.61, Tercil T2 en categoría \emph{Energy \& Fuels})

\item \textbf{O. Perpiñán}, E. Lorenzo, y M. A. Castro. On the calculation of energy produced by a PV grid-connected system. Progress in Photovoltaics: Research and Applications, 15(3):265–274, 2007.$\backslash$\ (Índice de impacto: 2.18; Tercil T1 en categoría \emph{Energy \& Fuels})
\end{enumerate}

\subsection{Publicaciones Nacionales}
\label{sec:orge3e2408}

\begin{itemize}
\item Fernando Garnacho Vecino, Miguel Ángel Sánchez-Urán González, Javier Ortego La Moneda, F. Alvarez, \textbf{O. Perpiñán}, Revisión periódica del estado del aislamiento de los cables de AT mediante medidas de DPs on line, Energía: Ingeniería energética y medioambiental, ISSN 0210-2056, Año nº 37, Nº 230, 2011, págs. 38-46.

\item \textbf{O. Perpiñán}, E. Lorenzo, M.A. Castro, Estimación de sombras mutuas entre seguidores y optimización de separaciones, Era Solar, ISSN 0212-4157, Nº. 131, 2006 , págs. 28-37

\item \textbf{O. Perpiñán}, M.A. Castro, E. Lorenzo, Análisis y comparación de funcionamiento de grandes plantas: Photocampa y Forum Energía: Ingeniería energética y medioambiental, ISSN 0210-2056, Año nº 32, Nº 190, 2006, págs. 63-68

\item J. Carretero, L. Mora-López, \textbf{O. Perpiñán}, A. Pereña, Mariano Sidrach de Cardona Ortín, I. Martínez, M. Aritio, Parque tecnológico de Andalucía: tecnología OPC. Monitorización wireless de una instalación fotovoltaica de 56 kWp, Era solar: Energías renovables, ISSN 0212-4157, Nº. 127, 2005, págs. 56-65.

\item Arancha Perpiñán Lamigueiro, \textbf{O. Perpiñán}, Elena Carmen Horno, Bombeo de agua para riego con energía solar fotovoltaica: sistemas de bombeo solar directo Riegos y drenajes XXI, ISSN 0213-3660, Nº 144, 2005, págs. 68-74

\item \textbf{O. Perpiñán}, R. Eyras, D. Jiménez, Antonio Gómez Avilés-Casco, Sistemas fotovoltaicos en el Parque de las Ciencias de Granada Era solar: Energías renovables, ISSN 0212-4157, Nº. 104, 2001, págs. 16-21
\end{itemize}


\section{Proyectos de Investigación Subvencionados}
\label{sec:org19d233b}
(Ordenados por fecha de inicio descendente)

\subsection{PVCROPS. PhotoVoltaic Cost reduction, Reliability, Operational performance, Prediction and Simulation}
\label{sec:orgae9090d}
\begin{itemize}
\item Entidad Financiadora: Comisión Europea
\item Cantidad Financiada (€): 5800352
\item Referencia del Proyecto: 308468
\item Tipo de convocatoria: UE
\item Entidades Participantes: UPM, UPNA, UEVORA, Cl Senes, Acciona, Ingeteam, ONE, Sources of the Bulgarian Academy of Sciences, DIT, Sunswitch SA, Rtnoe, Apere
\item Duración: desde 01/11/2012 hasta 31/10/2015 N° total de meses: 36
\item Investigador Principal: Luis Narvarte
\item Nº Investigadores: 10
\item Responsabilidad: Investigador colaborador
\item Aportación del Investigador:
\begin{itemize}
\item Predicción de productividad de sistemas fotovoltaicos a partir de modelos numéricos de predicción meteorológica y medidas terrestres.
\item Caracterización de las fluctuaciones en sistemas fotovoltaicos de gran tamaño en el contexto de su integración en las redes eléctricas convencionales.
\item Predicción de categorías de fluctuaciones en sistemas fotovoltaicos a partir de modelos numéricos de predicción meteorológica y medidas terrestres.
\item Desarrollo de dos paquetes software publicados con licencia GNU/GPL
\item Publicación de resultados en artículo JCR.
\end{itemize}
\end{itemize}

\subsection{ENERGOS: optimización de la cargabilidad de las líneas}
\label{sec:org1484810}

\begin{itemize}
\item Entidad Financiadora: CDTI
\item Cantidad Financiada (€): 380306
\item Referencia del Proyecto: CEN-2009-1048
\item Tipo de convocatoria: Nacional
\item Entidades Participantes: EUITI, Unión Fenosa
\item Duración: desde 07/07/2009 hasta 31/12/2012 N° total de meses: 42
\item Investigador Principal: Fernando Garnacho
\item Nº Investigadores: 8
\item Responsabilidad: Investigador colaborador
\item Aportación del Investigador:
\begin{itemize}
\item Revisión del estado del arte de sensores de alta frecuencia para Descargas Parciales en cables de MT.
\item Definición de parámetros característicos de Descargas Parciales en el contexto de su medida y clasificación en entornos industriales ruidosos.
\item Desarrollo de métodos de caracterización y clasificación no supervisada de colecciones de Descargas Parciales
\item Desarrollo de software pdCluster publicado con licencia GNU/GPL.
\item Publicación de resultados en artículo JCR.
\end{itemize}
\end{itemize}

\subsection{Caracterización de la variabilidad y comportamiento ante las perturbaciones de las plantas fotovoltaicas}
\label{sec:org2c9a9bb}

\begin{itemize}
\item Entidad Financiadora: Red Eléctrica de España
\item Tipo de convocatoria: Contrato con Empresa Privada
\item Entidades Participantes: UPM, UPNA, Acciona
\item Duración: desde 01/01/2009 hasta 31/12/2011 N° total de meses: 36
\item Investigador Principal: Eduardo Lorenzo
\item Nº Investigadores: 8
\item Responsabilidad: Investigador colaborador
\item Aportación del investigador:
\begin{itemize}
\item Caracterización de las fluctuaciones de radiación y potencia en sistemas fotovoltaicos mediante la transformada wavelet.
\item Análisis de las fluctuaciones de potencia en una planta fotovoltaica de gran tamaño según el tipo de día y la distancia entre generadores
\item Publicación de resultados en dos artículos JCR.
\end{itemize}
\end{itemize}

\subsection{Umbráculo Móvil}
\label{sec:org1e9f05c}

\begin{itemize}
\item Entidad Financiadora: Ministerio de Educación y Ciencia
\item Cantidad Financiada (€): 1469000
\item Referencia del Proyecto: PCT-120000-2007-34
\item Tipo de convocatoria: Nacional
\item Entidades Participantes: Isofotón, Jeronimo Vega Arquitectura, IES-UPM, Trim
\item Duración: desde 01/07/2007 hasta 31/12/2008 N° total de meses: 18
\item Investigador Principal: Oscar Perpiñán
\item Nº Investigadores: 15
\item Responsabilidad: Investigador principal
\item Aportación del Investigador: 
\begin{itemize}
\item Coordinación del proyecto
\item Desarrollo un seguidor horizontal adaptable a las particulares condiciones de una cubierta de un edificio.
\item Análisis de las condiciones de instalación y explotación de sistemas de seguimiento en zonas elevadas, frente a la experiencia de instalación en terreno común.
\item Diseño de sistema de conversión de potencia con inversores de gran tamaño.
\end{itemize}
\end{itemize}

\subsection{Desarrollo de una Plataforma para la Monitorización y Seguimiento de Sistemas Fotovoltaicos}
\label{sec:orgf3c9cb3}

\begin{itemize}
\item Entidad Financiadora: Corporación Tecnológica de Andalucia
\item Cantidad Financiada (€): 400034
\item Referencia del Proyecto: 06109D1A
\item Tipo de convocatoria: CC.AA.
\item Entidades Participantes: Isofotón, ISM, Universidad de Málaga
\item Duración: desde 01/01/2007 hasta 31/12/2007 N° total de meses: 12
\item Investigador Principal: Oscar Perpiñán
\item Nº Investigadores: 15
\item Responsabilidad: Investigador principal
\item Aportación del Investigador:
\begin{itemize}
\item Definición de requisitos del sistema.
\item Selección de equipos a monitorizar.
\item Diseño de sistemas de prueba.
\item Revisión de implementación de herramientas software.
\item Implementación de prototipos.
\item Estudio del estado del arte de comunicaciones en sistemas de monitorización.
\end{itemize}
\end{itemize}

\subsection{Desarrollo y Caracterización de Tejados y Fachadas Fotovoltaicas Ventiladas Integradas en Edificios}
\label{sec:org724aac0}

\begin{itemize}
\item Entidad Financiadora: Ministerio de Educación y Ciencia
\item Cantidad Financiada (€): 452289
\item Referencia del Proyecto: CIT-120000-2007-89
\item Tipo de convocatoria: Nacional
\item Entidades Participantes: CIMNE, Isofotón, Pich-Aguilera Arquitectos, Universidad de Lleida
\item Duración: desde 01/07/2007 hasta 30/06/2008 N° total de meses: 12
\item Investigador Principal: Jordi Cipriano
\item Nº Investigadores: 15
\item Responsabilidad: Investigador colaborador
\item Aportación del Investigador:
\begin{itemize}
\item Estudio del estado del arte de generadores fotovoltaicos en fachadas ventiladas.
\item Consultoría en la implementación de herramientas software de simulación con TRNSYS.
\item Definición del generador fotovoltaico y equipos asociados.
\item Diseño de prototipos para prueba en campo.
\item Análisis de datos de funcionamiento.
\end{itemize}
\end{itemize}

\subsection{Optimización del Diseño Eléctrico de Módulos Fotovoltaicos para Minimizar las Perdidas de Potencia por Dispersión y Evitar los Puntos Calientes}
\label{sec:orgb50d373}

\begin{itemize}
\item Entidad Financiadora: Ministerio de Industria, Turismo y Comercio
\item Cantidad Financiada (€): 338691
\item Referencia del Proyecto: FIT-030000-2007-264
\item Tipo de convocatoria: Nacional
\item Entidades Participantes: Isofotón, CIEMAT
\item Duración: desde 01/01/2007 hasta 31/12/2008 N° total de meses: 24
\item Investigador Principal: Paula Sanchez-Friera
\item Nº Investigadores: 6
\item Responsabilidad: Investigador colaborador
\item Aportación del Investigador:
\begin{itemize}
\item En el diseño y ejecución de sistemas fotovoltaicos de gran tamaño obra especial relevancia el impacto de sombras mutuas y las perdidas por dispersión, objeto de este proyecto de investigación. Dado que los cinco investigadores restantes son expertos en diseño y fabricación de dispositivos fotovoltaicos, la aportación del solicitante consistió en aportar el conocimiento de ingeniería desde la experiencia en campo.
\end{itemize}
\end{itemize}

\subsection{Conector de Paneles Dinámico CPD-1 y Convertidor Multipuente Multipotencia CMM-1}
\label{sec:org953347f}

\begin{itemize}
\item Entidad Financiadora: Ministerio de Industria, Turismo y Comercio
\item Cantidad Financiada (€): 354347
\item Referencia del Proyecto: FIT-120000-2007-100
\item Tipo de convocatoria: Nacional
\item Entidades Participantes: LACECAL, Isofotón, Calor Económico del Bierzo
\item Duración: desde 01/07/2007 hasta 30/06/2008 N° total de meses: 12
\item Investigador Principal: Jose Antonio Domínguez Vázquez (LACECAL)
\item Nº Investigadores: 12
\item Responsabilidad: Investigador colaborador
\item Aportación del Investigador:
\begin{itemize}
\item Diseño de prototipos para pruebas.
\item Diseño de sistema de monitorización.
\item Estudio del estado del arte de equipos inversores.
\end{itemize}
\end{itemize}

\subsection{MODEN II}
\label{sec:org9925c58}

\begin{itemize}
\item Entidad Financiadora: Ministerio de Industria, Turismo y Comercio
\item Cantidad Financiada (€): 391753
\item Referencia del Proyecto: FIT-120000-2006-138
\item Tipo de convocatoria: Nacional
\item Entidades Participantes: LACECAL, Isofotón, INGETEAM, Calor Económico del Bierzo
\item Duración: desde 01/01/2006 hasta 31/12/2007 N° total de meses: 24
\item Investigador Principal: Pablo Gutiérrez Martín (LACECAL)
\item Nº Investigadores: 13
\item Responsabilidad: Investigador colaborador
\item Aportación del Investigador:
\begin{itemize}
\item Diseño y ejecución de sistemas fotovoltaicos de conexión a red
\item Diseño de sistemas fotovoltaicos para probar los equipos diseñados
\item Coordinación de la participación de Isofotón en este proyecto
\end{itemize}
\end{itemize}

\subsection{Mejora de la Calidad del Servicio Energético en las Aplicaciones de Electrificación Rural}
\label{sec:org400966c}
\begin{itemize}
\item Entidad Financiadora: Ministerio de Educación y Ciencia
\item Cantidad Financiada (€): 356943
\item Referencia del Proyecto: CIT-120000-2005-75
\item Tipo de convocatoria: Nacional
\item Entidades participantes: Isofotón, IES-UPM, Universidad Internacional de Andalucia, Universidad de Santiago de Compostela
\item Duración: desde 01/01/2005 hasta 31/12/2005 N° total de meses: 12
\item Investigador Principal: Oscar Perpiñán
\item Nº Investigadores: 8
\item Responsabilidad: Investigador principal
\item Aportación del Investigador:
\begin{itemize}
\item Coordinación del proyecto.
\item Diseño de prototipo de sistema de bombeo de agua de gran tamaño acoplado a sistemas fotovoltaicos de seguimiento.
\item Desarrollo y validación de software de simulación de sistemas de bombeo fotovoltaico.
\item Supervisión del diseño de sistemas de riego con sistemas de bombeo.
\end{itemize}
\end{itemize}

\subsection{Caracterización del Comportamiento Térmico de la Fachada PVSKIN y su Interacción con Edificios Modelo en Clima Mediterráneo}
\label{sec:org1409861}

\begin{itemize}
\item Entidad Financiadora: Ministerio de Educación y Ciencia
\item Cantidad Financiada (€): 152520
\item Referencia del Proyecto: CIT-120000-2005-74
\item Tipo de convocatoria: Nacional
\item Entidades Participantes: Isofotón, CIMNE
\item Duración: desde 01/01/2005 hasta 15/05/2006 N° total de meses: 18
\item Investigador Principal: Ramón Eyras Daguerre
\item Nº Investigadores: 7
\item Responsabilidad: Investigador colaborador
\item Aportación del Investigador:
\begin{itemize}
\item Diseño de prototipos para medida.
\item Diseño de sistemas de monitorización para pruebas.
\item Coordinación de tareas de ingeniería.
\end{itemize}
\end{itemize}

\subsection{Sistema de Desalinización mediante Ósmosis Inversa Alimentado con Energía Solar Fotovoltaica}
\label{sec:orgbe361ee}

\begin{itemize}
\item Entidad Financiadora: CDTI
\item Cantidad Financiada (€): 626600
\item Referencia del Proyecto: 04-0624
\item Tipo de convocatoria: Nacional
\item Entidades Participantes: Isofotón, Instituto Tecnológico de Canarias, VEOLIA
\item Duración: desde 01/07/2004 hasta 31/12/2005 N° total de meses: 18
\item Investigador Principal: Ramón Eyras Daguerre
\item Nº Investigadores: 10
\item Responsabilidad: Investigador colaborador
\item Aportación del Investigador:
\begin{itemize}
\item Diseño de sistema fotovoltaico autónomo sin acumulación de energía que permite el funcionamiento autónomo de la planta en zonas aisladas sin suministro eléctrico.
\item Supervisión de la optimización de los circuitos hidráulicos, eléctricos y el sistema de control para el correcto y viable funcionamiento de la planta y el parque fotovoltaico.
\end{itemize}
\end{itemize}

\subsection{Grandes Centrales Fotovoltaicas}
\label{sec:org48e9932}

\begin{itemize}
\item Entidad Financiadora: Ministerio de Educación y Ciencia
\item Cantidad Financiada (€): 107690
\item Referencia del Proyecto: FIT-120000-2004-24
\item Tipo de convocatoria: Nacional
\item Entidades Participantes: Isofotón, IES-UPM, Universidad Internacional de Andalucía
\item Duración: desde 01/01/2004 hasta 31/12/2004 N° total de meses: 12
\item Investigador Principal: Oscar Perpiñán
\item Nº Investigadores: 8
\item Responsabilidad: Investigador principal
\item Aportación del Investigador:
\begin{itemize}
\item Coordinación de proyecto.
\item Análisis de datos de funcionamiento de plantas fotovoltaicas de gran tamaño.
\item Métodos de diseño en plantas fotovoltaicas de gran tamaño con tecnologías de seguimiento.
\item Análisis de métodos de cálculo de productividad en sistemas fotovoltaico con diferentes fuentes de radiación.
\item Desarrollo de herramienta software solaR.
\item Publicación de resultados en 4 artículos JCR.
\item Defensa de Tesis Doctoral
\end{itemize}
\end{itemize}

\subsection{Grandes Centrales Fotovoltaicas}
\label{sec:org47f8b2e}

\begin{itemize}
\item Entidad Financiadora: Ministerio de Educación
\item Cantidad Financiada (€): 313052
\item Referencia del Proyecto: CIT-120000-2005-68
\item Tipo de convocatoria: Nacional
\item Entidades Participantes: Isofotón, IES-UPM, Universidad Internacional de Andalucía
\item Duración: desde 01/01/2005 hasta 31/12/2005 N° total de meses: 12
\item Investigador Principal: Oscar Perpiñán
\item Nº Investigadores: 8
\item Responsabilidad: Investigador principal
\item Aportación del Investigador:
\begin{itemize}
\item Coordinación de proyecto.
\item Análisis de datos de funcionamiento de plantas fotovoltaicas de gran tamaño.
\item Métodos de diseño en plantas fotovoltaicas de gran tamaño con tecnologías de seguimiento.
\item Análisis de métodos de cálculo de productividad en sistemas fotovoltaico con diferentes fuentes de radiación.
\item Desarrollo de herramienta software solaR.
\item Publicación de resultados en 4 artículos JCR.
\item Defensa de Tesis Doctoral
\end{itemize}
\end{itemize}

\subsection{Combinación de Energías Renovables con Almacenamiento Intermedio de H2 y Pila de Combustible (TINA)}
\label{sec:orgf09c25a}
\begin{itemize}
\item Entidad Financiadora: Junta de Andalucía
\item Cantidad Financiada (€): 532358
\item Referencia del Proyecto: 2002000874
\item Tipo de convocatoria: CC.AA.
\item Entidades Participantes: Isofotón, BESEL, David Fuel
\item Duración: desde 02/01/2004 hasta 30/06/2005 N° total de meses: 18
\item Investigador Principal: Jesús Alonso
\item Nº Investigadores: 6
\item Responsabilidad: Investigador colaborador
\item Aportación del Investigador:
\begin{itemize}
\item Diseño del sistema fotovoltaico.
\item Supervisión de la instalación del prototipo.
\item Diseño del sistema de monitorización.
\item Análisis de resultados de funcionamiento.
\end{itemize}
\end{itemize}

\subsection{Heliodomo: Nuevo Concepto de Vivienda Autosuficiente}
\label{sec:orgcdaa66e}

\begin{itemize}
\item Entidad Financiadora: Ministerio de Educación y Ciencia
\item Cantidad Financiada (€): 205165
\item Referencia del Proyecto: BIA2004-05234
\item Tipo de convocatoria: Nacional
\item Entidades Participantes: IES-UPM, CEDINT, Isofoton
\item Duración: desde 13/12/2004 hasta 12/12/2007 N° total de meses: 36
\item Investigador Principal: F. Javier Neila González
\item Nº Investigadores: 15
\item Responsabilidad: Investigador colaborador
\item Aportación del Investigador:
\begin{itemize}
\item Diseño de sistemas fotovoltaicos en condiciones de integración arquitectónica
\item Análisis del funcionamiento de sistemas fotovoltaicos en condiciones de integración arquitectónica
\item Selección de materiales apropiados por su sinergia con generadores fotovoltaicos
\end{itemize}
\end{itemize}

\subsection{Sevilla PV}
\label{sec:orgc7d753c}

\begin{itemize}
\item Entidad Financiadora: Comisión Europea
\item Cantidad Financiada (€): 2759244
\item Referencia del Proyecto: NNE5-2001-00767
\item Tipo de convocatoria: UE
\item Entidades Participantes: Solucar, Solartec, CIEMAT, BP Solar, Saint-Gobain, Atersa, IDAE, Isofoton, WIP
\item Duración: desde 01/02/2004 hasta 31/07/2007 N° total de meses: 54
\item Investigador Principal: Rafael Osuna
\item Nº Investigadores: 50
\item Responsabilidad: Investigador colaborador
\item Aportación del Investigador:
\begin{itemize}
\item Diseño de sistemas fotovoltaicos con módulos fotovoltaicos con sistema externo de concentración 2x.
\end{itemize}
\end{itemize}
\subsection{PV Generators Integrated into Sound Barriers}
\label{sec:org0e71567}

\begin{itemize}
\item Entidad Financiadora: Comisión Europea
\item Cantidad Financiada (€): 4950039
\item Referencia del Proyecto: NNE5/2000/397
\item Tipo de convocatoria: UE
\item Entidades Participantes: Isofoton, Ayto. Helmond, Ayto. Leganés, Fraunhofer Ise, Biohaus
\item Duración: desde 01/01/2001 hasta 31/12/2003 N° total de meses: 24
\item Investigador Principal: Ramón Eyras Daguerre
\item Nº Investigadores: 25
\item Participación: Investigador colaborador
\item Aportación del Investigador:
\begin{itemize}
\item Supervisión del desarrollo de un módulo fotovoltaico sobre cubierta cerámica para atenuar ruido.
\item Supervisión de pruebas de funcionamiento del módulo desarrollado en probeta acústica.
\item Diseño de sistema fotovoltaico de gran tamaño en talud de autovía.
\end{itemize}
\end{itemize}

\subsection{PV Grid Connected in a Car Parking}
\label{sec:org149fdb3}

\begin{itemize}
\item Entidad Financiadora: Comisión Europea
\item Cantidad Financiada (€): 2612600
\item Referencia del Proyecto: NNE5/1999/772
\item Tipo de convocatoria: UE
\item Entidades Participantes:, ISOFOTON, BERGE Y CIA, Universidad de Northumbria, Biohaus, ICAEN, Sunwatt
\item Duración: desde 01/01/2000 hasta 31/12/2002 N° total de meses: 24
\item Investigador Principal: Ramon Eyras Daguerre
\item Nº Investigadores: 20
\item Responsabilidad: Investigador colaborador
\item Aportación del Investigador:
\begin{itemize}
\item Diseño de un sistema fotovoltaico de gran tamaño integrado en una estructura de aparcamiento.
\item Estudio del arte de sistemas de conversión de potencia.
\item Diseño de sistemas de conversión de potencia variados para comparativa.
\item Diseño de sistema de monitorización para investigación.
\item Análisis de datos de funcionamiento. Modelado y validación de modelos.
\item Publicación de resultados en artículos JCR
\item Estudio del estado del arte de sistemas de seguridad eléctrica en sistemas fotovoltaicos con público.
\end{itemize}
\end{itemize}


\section{Comunicaciones y Ponencias Presentadas a Congresos}
\label{sec:org5c12563}
\subsection{31st European Photovoltaic Solar Energy Conference and Exhibition}
\label{sec:org827afd9}
Congreso Internacional organizado por WIP. Septiembre 2015.

\begin{itemize}
\item Comparative Study of Nonparametric and Parametric PV Models to Forecast AC Power Output of PV Plants, M.P. Almeida, M. Muñoz, I. de la Parra, \textbf{O. Perpiñán}, L. Narvarte, 10.4229/EUPVSEC20152015-5BV.2.16

\item Using a Nonparametric PV Model to Forecast AC Power Output of PV Plants, M.P. Almeida, \textbf{O. Perpiñán}, L. Narvarte, 10.4229/EUPVSEC20152015-5BV.2.18.
\end{itemize}

\subsection{EnerSol World Sustainable Energy Forum 2014}
\label{sec:org105d544}
Congreso Internacional organizado por SETCOR. Túnez, Noviembre 2014.

\begin{itemize}
\item Linke turbidity prediction for improving solar radiation forecasting, F. Antonanzas-Torres, F. J. Martinez-de-Pison, \textbf{O. Perpiñán}, R. Nunes, Carlos Coimbra
\end{itemize}

\subsection{VI Jornadas de Usuarios de R}
\label{sec:org0328770}
Congreso Nacional organizado por la Comunidad R-Hispano. Santiago de Compostela, Octubre 2014. 

\begin{itemize}
\item meteoForecast: predicciones meteorológicas de modelos NWP en R, \textbf{O. Perpiñán}, M. P. Almeida
\end{itemize}

\subsection{17th International Congress on  Project Management and Engineering}
\label{sec:org9f06dcc}
Congreso Internacional organizado por AEIPRO. Málaga, Julio 2013

\begin{itemize}
\item Downscaling of Solar Irradiation from Satellite Models, F. Antoñanzas,J. Antoñanzas, F.J. Martínez de Pisón, M. J. Alía Martínez, \textbf{Perpiñán, O.}
\end{itemize}
\subsection{CIGRE SESSION 2012}
\label{sec:orgf6752d9}
Congreso Internacional organizado por CIGRE. Francia, Agosto 2012

\begin{itemize}
\item New Procedure To Determine Insulation Condition Of High Voltage Equipment By Means Of PD Measurements In Service, F. Garnacho, M.A. Sánchez-Urán, J. Ortego, F. Álvarez, \textbf{O. Perpiñán}, E. Puelles, R. Moreno, D. Prieto, D. Ramos
\end{itemize}

\subsection{III Jornadas de usuarios de R}
\label{sec:org00720e5}
Congreso Nacional organizado por la Comunidad R-Hispano. Madrid, Noviembre 2011

\begin{itemize}
\item Comparativa y análisis de variabilidad espacial entre medidas de radiación solar terrestre y satelital, F. Antoñanzas, F. Cañizares, R. Morales, M. Ojeda, \textbf{O. Perpiñán}

\item solaR: geometría, radiación y energía solar en R, \textbf{O. Perpiñán}

\item Datos geográficos de tipo raster en R, J. van Etten, \textbf{O. Perpiñán}, R. J. Hijmans
\end{itemize}

\subsection{JICABLE 2011}
\label{sec:org83eba79}
Congreso Internacional organizado por Jicable. Francia, Junio 2011

\begin{itemize}
\item PD monitoring system of HV cable, F. Garnacho, M.A. Sánchez-Urán, J. Ortego, J. Moreno, F. Álvarez, \textbf{O. Perpiñán}
\end{itemize}

\subsection{22nd European Photovoltaic Solar Energy Conference}
\label{sec:org43d5175}
Congreso Internacional organizado por WIP. Milán, Septiembre 2007

\begin{itemize}
\item PV Solar Tracking Systems Analysis, R. Sorichetti, \textbf{O. Perpiñán}
\end{itemize}

\subsection{21st European Photovoltaic Solar Energy Conference}
\label{sec:org7ca2b66}
Congreso Internacional organizado por WIP. Dresden, Septiembre 2006

\begin{itemize}
\item A Real Case Of Building Integrated PV, Isofotón Offices In Málaga, F. Arribas, I. Eyras, J. Vega, L. Mendez, J.J. Garcia, \textbf{O. Perpiñán}, R. Eyras
\end{itemize}

\subsection{9th International Congress on  Project Management and Engineering}
\label{sec:org552605c}
Congreso Internacional organizado por AEIPRO. Málaga, Junio de 2005

\begin{itemize}
\item Instalación De Energía Solar En La Nueva Fabrica De Isofotón En El P.T.A De Málaga, L. Mendez, J. Vega, J.J. Garcia, I. Eyras, F. Arribas, \textbf{O. Perpiñán}, R. Eyras
\end{itemize}

\subsection{20th European Photovoltaic Solar Energy Conference}
\label{sec:org626e746}
Congreso Internacional organizado por WIP. Barcelona, Junio 2005

\begin{itemize}
\item Analysis And Comparison Of Performance Of Large Plantes: Photocampa And Forum, \textbf{O. Perpiñán}, E. Lorenzo, M.A. Castro, R. Eyras
\end{itemize}

\subsection{XII Congreso Ibérico e VII Congresso Ibero-Americano de Energía Solar}
\label{sec:org32490db}
Congreso Nacional organizado por AEDES-ISES. Vigo, Septiembre 2004

\begin{itemize}
\item Monitorización Wireless De Instalación Fotovoltaica De 56 kWp En El Parque Tecnológico De Andalucia Basada En La Tecnologia OPC , M. Sidrach, J. Carretero, A. Pereña, L. Mora, M. Aritio, \textbf{O. Perpiñán}

\item Centrales Hibridas Solar-Diesel: Nuestra Experiencia, M. Mazzorana, L. Carrasco, E. Horno, R. Eyras, \textbf{O. Perpiñán}, L.  Narvarte

\item Sistema Solar Térmico Y Fotovoltaico En Hotel Urbano, J. Garcia, \textbf{O. Perpiñán}, F. Ramirez, R. Eyras, J. Vega

\item Experiencia En Sistemas De Bombeo Solar Y Simulación Matemática De Bombeos Solares Con Equipos Estándar , E. Horno, \textbf{O. Perpiñán}, J. Hungria, I. Rai, R. Eyras

\item Fachada Doble Fotovoltaica PVskin: Prototipos, Investigación Y Desarrollo, I. Eyras, F. Arribas, M.A. Bofill, J. Vega, \textbf{O. Perpiñán}

\item Solarízate: Proyecto Escuelas Solares De Greenpeace-Idae, A. Gonzalez, \textbf{O. Perpiñán}, F. Ramirez, R. Eyras
\end{itemize}

\subsection{19th European Photovoltaic Solar Energy Conference}
\label{sec:orgc8a24a0}
Congreso Internacional organizado por WIP. Paris, Junio 2004

\begin{itemize}
\item PV Soundless: World Record Along The Highway -- A PV Sound Barrier With 500 kWp And Ceramic Based PV Modules, M. Grottke, T. Suker, R.Eyras, J.Gorbeña, \textbf{O. Perpiñán}, A. Voigt, A.  Thiel, M.Spendel, K.Gherlicher, G. Frisen, R.Gambi, K.Kellner

\item Architecture And PV - Discussion And Experiences, J. Vega, \textbf{O. Perpiñán}

\item Forum Solar: A Large PV Pergola For Forum 2004, \textbf{O. Perpiñán}, A. Gonzalez, I. Eyras, R. Eyras

\item PV Pumping Systems For Domestic Water Supply-Cases Of Study, R. Eyras, \textbf{O. Perpiñán}, J. Hungria, I. Rai
\end{itemize}

\subsection{7th Congreso Internacional de Ingenieria de Proyectos}
\label{sec:orgf085eae}
Congreso Internacional organizado por AEIPRO. Pamplona, Octubre 2003

\begin{itemize}
\item Instalación De Energía Solar Térmica Con Maquina De Absorción, J. Giacardi, I. Eyras, J.J. Garcia, \textbf{O. Perpiñán}, R. Eyras

\item Ósmosis Inversa Alimentada Mediante Energía Solar Fotovoltaica, \textbf{O. Perpiñán}, M. Aritio, J. Hungria, T. Espino, L. Guerrero, J.M. Ortega
\end{itemize}

\subsection{3rd World Conference on Photovoltaic Energy Conversion}
\label{sec:orgc7328af}
Congreso Internacional organizado por IEEE. Osaka, Octubre 2003

\begin{itemize}
\item Photocampa: Design And Performance Of The PV System, \textbf{O. Perpiñán}, N. Pearsall, L. Mendez, R. Eyras

\item Architectural Integration Of Grid Connected Photovoltaic Systems For Schools In Coslada, D. Jimenez, L. M. Carrasco, R. Eyras, \textbf{O. Perpiñán}, A. Gonzalez
\end{itemize}

\subsection{5th European Conference on Noise Control}
\label{sec:orgc169af5}
Congreso Internacional organizado por CNR-Instituto di Acustica. Napoles, Mayo 2003

\begin{itemize}
\item Photovoltaic Modules Integrated In Novel Noise Barrier Elements, T. Erge, R. Eyras, \textbf{O. Perpiñán}, A. Gonzalez, E. Rossler
\end{itemize}

\subsection{6th Congreso Internacional de Ingeniería de Proyectos}
\label{sec:orgb4b33f4}
Congreso Internacional organizado por AEIPRO. Barcelona, Octubre 2002

\begin{itemize}
\item Photocampa: Sistema Fotovoltaico Integrado En Estructura De Aparcamiento, \textbf{O. Perpiñán}, S. Izquierdo, L. Mendez, S. Salat, R. Eyras
\end{itemize}

\subsection{PV in Europe}
\label{sec:org1a1cfc8}
Congreso Internacional organizado por WIP. Roma, Octubre 2002

\begin{itemize}
\item PVSOUNDLESS: Large PV Sound Barrier Along A Railway, A. Gonzalez, R. Eyras, \textbf{O. Perpiñán}, T. Erge

\item FIVE Project. Integration Of PV Systems On Health Emergency Vehicles. Results And Conclusions, G. Almonacid, F.J. Muñoz, J. De La Casa, J. C. Hernandez, J. De La Casa Cardenas, J.D. Aguilar, P. Serrano, A. Mantero, A. Jimenez, E.  Ferrando, \textbf{O. Perpiñán}, R. Eyras,
\end{itemize}

\subsection{EuroSun 2002}
\label{sec:org3596403}
Congreso Internacional organizado por ISES. Bologna, Junio 2002

\begin{itemize}
\item PVSOUNDLESS: large PV sound barrier along a railway, A.G. Marsiñach, O.Perpiñán, T.Erge, R.Eyras
\end{itemize}

\subsection{17th European Photovoltaic Solar Energy Conference}
\label{sec:orgfd60164}
Congreso Internacional organizado por WIP. Munich, Octubre 2001

\begin{itemize}
\item Photocampa: PV System Integrated Into A Large Car Park, \textbf{O. Perpiñán}, N. Pearsall, L. Mendez, W. Ernst, M. Schneider, R. Eyras

\item Special Module Types For PV Systems In High-Profile Buildings, \textbf{O. Perpiñán}, J. Vega, R. Eyras

\item PV Pergola For The Chapel Of Men, Guayasamin Foundation, Ecuador, R. Eyras, Y. Fernandez, \textbf{O. Perpiñán}

\item Integration Of PV Systems On Health Emergency Vehicles, G. Almonacid, F. J. Muñoz, J. De La Casa, J.C. Hernandez, J. De La Casa Cardenas, P. Serrano, A. Mantero, A. Jimenez, E. Ferrando, O.  Perpiñán, R. Eyras

\item Quality Control of Luminaries in SHS, L. Narvarte, E. Lorenzo, \textbf{O. Perpiñán}, M.A. Egido
\end{itemize}


\section{Desarrollos}
\label{sec:org3024ec1}
\subsection{solaR}
\label{sec:org3ff6970}

Paquete software basado en \texttt{R} compuesto por un conjunto de funciones destinadas al cálculo de la radiación solar incidente en sistemas fotovoltaicos y a la simulación del funcionamiento de diferentes aplicaciones de esta tecnología energética. En la versión actual de este paquete se incluyen funciones que permiten realizar todas las etapas de cálculo desde la radiación global en el plano horizontal hasta la productividad final de sistemas fotovoltaicos de conexión a red y de bombeo. Esta herramienta se publica con una licencia libre GNU/GPL en el \href{http://cran.r-project.org/web/packages/solaR/index.html}{Comprehensive R Archive Network (CRAN)} y en la web \url{http://oscarperpinan.github.io/solar/}.

\subsection{rasterVis}
\label{sec:org97b2155}
Paquete software basado en \texttt{R} para la visualización e interacción gráfica de datos espaciales masivos. Esta herramienta se publica con una licencia libre GNU/GPL en el \href{http://cran.r-project.org/web/packages/rasterVis/index.html}{Comprehensive R Archive Network (CRAN)} y en su página web \url{http://oscarperpinan.github.io/rastervis}.

\subsection{meteoForecast}
\label{sec:org069b0a2}
Paquete software basado en \texttt{R} que permite obtener predicciones de modelos numéricos meteorológicos producidos por diferentes servicios en formato raster o como series temporales. Esta herramienta se publica con una licencia libre GNU/GPL en el \href{http://cran.r-project.org/web/packages/meteoForecast/index.html}{Comprehensive R Archive Network (CRAN)} y en su página web \url{https://github.com/oscarperpinan/meteoForecast/}.

Este desarrollo software es uno de los resultados del proyecto europeo PVCROPS. Ha sido empleado en el artículo M. Pinho Almeida, O. Perpiñán, L. Narvarte, \guillemotleft{}PV Power Forecast Using a Nonparametric PV Model Solar Energy\guillemotright{} Solar Energy, 2015. 

\subsection{PVF}
\label{sec:org4e5d66d}
Paquete software basado en \texttt{R} que permite realizar predicciones de potencia producida por un sistema FV. Esta herramienta se publica con una licencia libre GNU/GPL en la página web \url{http://github.com/iesiee/PVF}.

Este desarrollo software es uno de los resultados del proyecto europeo PVCROPS. Ha sido empleado en el artículo M. Pinho Almeida, O. Perpiñán, L. Narvarte, \guillemotleft{}PV Power Forecast Using a Nonparametric PV Model Solar Energy\guillemotright{} Solar Energy, 2015. 

\subsection{pdCluster}
\label{sec:org6fd73df}
Paquete software basado en \texttt{R} para la cuantificación, clasificación y análisis de importancia de variables de señales de descargas parciales en equipos de Alta Tensión. Esta herramienta se publica con una licencia libre GNU/GPL. Está accesible desde la página web \url{http://oscarperpinan.github.io/pdcluster/}.

Este desarrollo software es uno de los resultados del proyecto CENIT ENERGOS. Ha sido empleado en el artículo O. Perpiñán,
M.A. Sánchez-Urán, F. Álvarez, J. Ortego, F. Garnacho, Signal analysis and feature generation for pattern identification of
partial discharges in high-voltage equipment, Electric Power Systems Research, 2013, 95:C (56-65).


\section{Cursos y Seminarios Impartidos}
\label{sec:org736ab01}
\begin{itemize}
\item Curso \guillemotleft{}Introducción a R\guillemotright{} (8 horas) para profesores de la UPM (Instituto de Ciencias de la Educación, UPM):  Noviembre 2017, Enero 2017, Enero 2016, Diciembre 2014.

\item Curso \guillemotleft{}R avanzado\guillemotright{} (8 horas) para profesores de la UPM (Instituto de Ciencias de la Educación, UPM): Junio 2018, Julio 2017, Febrero 2016.

\item Curso \guillemotleft{}Introducción a R\guillemotright{} (10 horas) para alumnos del CEIGRAM-UPM: Septiembre 2015, Noviembre 2014.

\item Taller \guillemotleft{}Visualización de Series Temporales\guillemotright{} en las VI Jornadas de Usuarios de R (Santiago de Compostela, Octubre 2014)

\item Taller \guillemotleft{}Visualización de Datos Raster\guillemotright{} en las VI Jornadas de Usuarios de R (Santiago de Compostela, Octubre 2014)

\item Curso \guillemotleft{}Introducción a R\guillemotright{} (8 horas) para profesores de la UNED (UNED, Marzo 2013).

\item Curso \guillemotleft{}R avanzado\guillemotright{} (8 horas) para profesores de la UNED (UNED, Marzo 2013).

\item Ponencia \guillemotleft{}Data Visualization with R\guillemotright{} dentro del Máster \guillemotleft{}Data Driven Methods in Environmental Management and Conservation\guillemotright{} del (Instituto de Empresa, Febrero 2013).

\item Participación en las ediciones 2014/2015, 2013/2014, 2012/2013, 2011/2012, 2010/2011 y 2010/2009 del Master propio de Energías Renovables y Medio Ambiente de la UPM, organizado por la ETSIDI-UPM, impartiendo el tema ``Diseño de plantas FV con seguimiento solar'' con una duración de 4,5 horas.

\item Curso ``Instalaciones de energía solar'', impartido del 15/09/10 al 16/10/11 con una duración de 109 horas, organizado por la ETSI-Montes-UPM, impartiendo el modulo ``Sistemas fotovoltaicos conectados a red'', con una duración de 5 horas.

\item Formación a distancia sobre Diseño y Optimización de Sistemas Fotovoltaicos al responsable de Sistemas Solares de la empresa MENA. Este proceso de formación, con una duración de 6 meses, se ha basado en las potencialidades del paquete software solaR, reseñado anteriormente.

\item Curso ``Técnico en energías renovables'', impartido del 01/09/09 al 27/10/09 con una duración de 200 horas, organizado por la ETSIDI-UPM, impartiendo el modulo ``Sistemas fotovoltaicos conectados a red'', con una duración de 5 horas.

\item Curso ``Instalaciones de energía solar'', impartido del 15/09/09 al 16/10/09 con una duración de 109 horas, organizado por la ETSI-Montes-UPM, impartiendo el modulo ``Sistemas fotovoltaicos conectados a red'', con una duración de 5 horas.

\item Curso ``Técnico en instalaciones solares en edificios'', impartido del 01/09/09 al 15/10/09, organizado por la ETSIDI-UPM, impartiendo el módulo ``Sistemas fotovoltaicos conectados a red'', con una duración de 25 horas.

\item Participación en el curso ``Técnico en instalaciones fotovoltaicas y eólicas'', impartido del 06/10/09 al 04/12/09, organizado por la ETSIDI-UPM, impartiendo el módulo ``Sistemas fotovoltaicos conectados a red'', con una duración de 15 horas.

\item Participación en el curso ``Energías renovables'', ediciones 2011/2012, 2010/2011 y 2009/2010, con una duración de 200 horas, organizado por la ETSI-Montes-UPM, impartiendo el módulo ``Energía solar fotovoltaica'', con una duración de 5 horas.
\end{itemize}

\section{Cursos y Seminarios recibidos}
\label{sec:org08b43ee}
\emph{(Ordenados por duración)}

\begin{itemize}
\item Experto Universitario en Métodos Avanzados de Estadística Aplicada (UNED, 2009/2010, 625 horas)

\item Aplicación de las energías renovables (ETSII-UPC, 2001/2002, 300 horas)

\item Caracterización de la radiación solar como recurso energético (CIEMAT, 2006, 30 horas)

\item Estadística en la investigación experimental (ICE-UPM, 2010, 28 horas)

\item Estadística comparativa y de investigación para uno y dos grupos de muestras (ICE-UPM, 2009, 24 horas)

\item Introduction to mathematical optimization techniques applied to power systems generation operation planning (Universidad Pontificia de Comillas, 2003, 20 horas)

\item Curso eléctrico de Media Tensión (Pedro Giner Editorial, 2003, 18 horas)

\item Rechargeable batteries (OTTI Kolleg, 2002, 20 horas)

\item Wavelets en Estadística (ICE-UPM, 2010, 8 horas)
\end{itemize}


\section{Becas, Ayudas y Premios recibidos}
\label{sec:orgd5923d0}
\begin{itemize}
\item Premio Extraordinario de Doctorado.
\item Beca de Colaboración dentro del Grupo de Bioingeniería y Telemedicina de la ETSIT en el contexto del proyecto europeo \guillemotleft{}Worldwide Emergency Telemedicine Services\guillemotright{} (WETS) (1998)
\end{itemize}


\section{Actividad en Empresas y Profesión Libre}
\label{sec:org8af1b5a}
\begin{itemize}
\item Octubre 2010- Enero 2009: Ejercicio libre de la profesión:
Consultoría sobre diseño y análisis de funcionamiento de sistemas fotovoltaicos.

\item Diciembre 2008- Enero 2007: Subdirector Técnico de ISOFOTON
\begin{itemize}
\item Equipo compuesto por 10 personas.
\item Responsable de las Áreas de:
\begin{description}
\item[{I+D+i}] desarrollo de nuevas aplicaciones, coordinación y participación en proyectos colaborativos de I+D+i
\item[{Producto BOS}] desarrollo de equipos electrónicos para diferentes aplicaciones, supervisión de los equipos disponibles en el mercado
\item[{Difusión Técnica}] elaboración de documentación, formación de técnicos e ingenieros, participación en conferencias y cursos de especialización
\end{description}
\end{itemize}

\item Enero 2007-Mayo 2002: Gerente de Ingeniería (Dpto. Técnico de ISOFOTON)
\begin{itemize}
\item Responsable de ofertas técnicas, diseño de proyectos y proyectos de ejecución.
\item Equipo compuesto por 7 personas.
\end{itemize}

\item Mayo 2002-Noviembre 2001: Gerente de Instalaciones (Dpto. Técnico de ISOFOTON)
\begin{itemize}
\item Responsable de gestión y dirección de proyectos, y jefatura de obras.
\item Equipo compuesto por 7 personas.
\end{itemize}

\item Noviembre 2001-Marzo 2000: Ingeniero de Proyectos (Dpto. Técnico de ISOFOTON)
\end{itemize}



\section{Otros Méritos Docentes o de Investigación}
\label{sec:org7da6037}
\subsection{Acreditaciones}
\label{sec:org34be837}

\begin{itemize}
\item Acreditación de la Agencia Nacional de Evaluación de la Calidad y Acreditación (ANECA) para la figura de Profesor Titular de Universidad.
\end{itemize}

\begin{itemize}
\item Evaluación favorable de la actividad docente según el programa DOCENTIA de la ANECA durante el período comprendido entre noviembre de 2008 y junio de 2011.
\end{itemize}

\subsection{Miembro de tribunal de Tesis Doctoral}
\label{sec:orgdc0fb57}

\begin{itemize}
\item \guillemotleft{}Estimation and forecasting methods for design and operation of photovoltaic plants\guillemotright{} (Javier Antoñanzas Torres, U. de la Rioja 2018)

\item ``El Proyecto Pierre Auger como Red de Sistemas Fotovoltaicos Aislados de Alta Estadística'' (Iago Rodriguez Cabo, USC 2015)

\item ``Predicción espacio-temporal de la irradiancia solar global a corto plazo en España mediante geoestadística y redes neuronales artificiales'' (Federico Vladimir Gutierrez Corea, UPM 2014)

\item ``Energía solar fotovoltaica: competitividad y evaluación económica, comparativa y modelos'' (Eduardo Collado Fernández, UNED 2009)
\end{itemize}

\subsection{Revisor para revistas}
\label{sec:org2735cee}

\begin{itemize}
\item \emph{Solar Energy}

\item \emph{Applied Energy}

\item \emph{Journal of Solar Engineering}

\item \emph{Atmospheric Measurement Techniques} and \emph{Atmospheric Chemistry and Physics} (EGU)

\item \emph{Computers and Geosciences}

\item \emph{IET Renewable Power Generation}

\item \emph{Journal of Statistical Software}
\end{itemize}

\subsection{Asociaciones}
\label{sec:org9f704a5}

\begin{itemize}
\item Co-presidente del Comité Científico de las jornadas CIES 2018 (\url{http://cies-congreso.org/15227/section/9534/xvi-congreso-iberico-y-xii-congreso-iberoamericano-de-energia-solar.html})

\item Presidente del Comité Organizador y miembro del Comité Científico de las III Jornadas de Usuarios de R (\url{http://r-es.org/III+Jornadas}).

\item Miembro del Comité Científico de las IV, V, y VI Jornadas de Usuarios de R (\url{http://r-es.org/IV+Jornadas}, \url{http://r-es.org/V+Jornadas}, \url{http://r-es.org/VI+Jornadas}).

\item Vocal de la Asociación de Usuarios de R.

\item Miembro de la Comisión Técnica de la Asociación de la Industria Fotovoltaica (ASIF) hasta Diciembre del 2008.

\item Miembro del grupo GT C del Comité de Normalización SC82 hasta Diciembre del 2008.
\end{itemize}
\end{document}